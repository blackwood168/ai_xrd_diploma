\newpage
\section{Заключение}
\sloppy{
Результаты данной работы показывают, что радиационная стойкость бензола, наблюдаемая в конденсированных средах, не определяется непосредственно молекулярными свойствами бензола, а является, в первую очередь, следствием эффективной диссипации энергии в конденсированных средах за счёт образования эксимеров бензола. Показано, что изолированные  молекулы бензола достаточно эффективно разлагаются в результате радиолиза в аргоновой матрице: в этих условиях значения радиационно-химических выходов разложения бензола в матрицах сопоставимы с выходами разложения других органических молекул в инертных матрицах. При этом характеристики матрицы (в частности, поляризуемость) оказывают большое влияние на радиационно-химический выход разложения бензола: в матрице аргона бензол расходуется при облучении примерно в 6 раз эффективнее, чем в более тяжёлых матрицах (Kr, Xe).

Показано, что основными каналами радиационно-индуцированных превращений бензола в матрицах инертных газов являются распад на фенильный радикал и атом водорода, а также изомеризация в фульвен. Характеристики матрицы (поляризуемость, атомный номер) существенно влияют на соотношение этих каналов. Отношение выходов фенильного радикала и фульвена резко возрастает в ряду Ar<Kr<Xe. Кроме того, зафиксировано образование {\itshape цис}- и {\itshape транс}-гексадиен\nobreakdash-1,3\nobreakdash-ина\nobreakdash-5, которые, вероятно, возникают в результате превращений возбуждённых молекул фульвена. Анализ дозовых зависимостей (кривых накопления) этих изомеров в различных матрицах показывает, что они могут возникать как непосредственно из бензола (без промежуточной стабилизации фульвена, образующегося предположительно в возбуждённом состоянии), так и в результате вторичного распада молекул фульвена, стабилизированных в матрицах.   

Впервые исследован радиолиз и фотолиз C$_6$D$_6$ в матрицах инертных газов. Показано, что дейтерирование не оказывает принципиального влияния на основные каналы радиолиза бензола и их соотношение. Основные тенденции, возникающие при радиолизе C$_6$H$_6$, сохраняются и в случае C$_6$D$_6$. Отсутствие влияния дейтерирования на основные каналы радиолиза, по-видимому, говорит о сравнительно малом вкладе процессов, протекающих через колебательно возбуждённые состояния, в образование основных первичных продуктов.   Впервые получены ИК-спектроскопические характеристики дейтерированных изомеров бензола (фульвена-$d_6$, бензвалена-$d_6$, бензола Дьюара-$d_6$).

Образующийся при радиолизе фенильный радикал является одним из ключевых интермедиатов  в предполагаемых механизмах образования ПАУ в космическом пространстве. Демонстрация эффективного образования этого радикала при радиолизе бензола (в отличие от фотолиза) в низкотемпературных матрицах может служить первым шагом на пути к экспериментальной верификации данных механизмов. Следует отметить, что состав продуктов при фотолизе и радиолизе бензола в матрицах различен из-за принципиально различных механизмов передачи энергии. Это обстоятельство необходимо учитывать при построении и верификации механизмов «холодного» синтеза ПАУ в межзвёздной среде.

В целом, сравнительное исследование механизмов радиационно-индуцированных превращений изолированных молекул бензола и его изотопологов при рентгеновском облучении в различных низкотемпературных инертных матрицах с использованием метода ИК-спектроскопии,  проведённое в данной работе, позволило получить принципиально новую информацию, которая может быть важна не только для астрохимии, но и для фундаментальной радиационной химии, фотохимии и спектроскопии.

}