\newpage
\section{обзор}
\section{Обзор литературы}
\sloppy{
В данном обзоре проведён анализ имеющихся работ по изучению радиолиза и фотолиза бензола, изложены основы метода матричной изоляции и особенности радиолиза в этих условиях.

\subsection{Метод матричной изоляции}
\label{isolation}
Изучение строения и свойств высокореакционноспособных частиц, таких как, например, радикалы и ион-радикалы, представляет собой нетривиальную экспериментальную задачу.
Одним из подходов к её решению является применение метода матричной изоляции~--- замораживания молекул исследуемого вещества в жёстком окружении (матрице) химически инертного вещества.
Кроме того, применение данного метода позволяет изучать <<молекулярные>> свойства вещества, которые обусловлены только свойствами отдельной молекулы
и не связаны с взаимодействием с окружением в конденсированной фазе.
Для его реализации необходимо создать малую концентрацию исследуемого вещества и выбрать в качестве матрицы инертное в условиях эксперимента вещество.
Типичные для матричной изоляции разбавления составляют от 1:1000 до 1:10000 (иногда применяют разбавление 1:100). Уже при соотношении 1:1000 достигается изоляция почти 99\%
слабовзаимодействующих молекул \cite{MI}.

При выборе матричного вещества руководствуются следующими требованиями:
\begin{itemize}
\item химическая инертность по отношению к исследуемому веществу и продуктам его реакций;
\item жёсткость, препятствующая взаимодействию изолированных частиц;
\item низкая поляризуемость;
\item прозрачность в необходимом диапазоне электромагнитного спектра или отсутствие магнитных ядер (при применении ЭПР спектроскопии).
\end{itemize}

Перечисленным требованиям удовлетворяют матрицы твёрдых инертных газов (неон, аргон, криптон, ксенон) \cite{MI}. Неон обладает самой низкой поляризуемостью, меньше всего возмущает колебательную структуру изолированной частицы,
а значит лучше всего подходит для определения спектроскопических характеристик исследуемых веществ. Однако его применение сильно ограничено его малым температурным 
диапазоном стабильности (10--11~K).   Аргоновые матрицы наиболее распространены из-за малой поляризуемости и температурного 
диапазона стабильности до 38--40~K. Неон и аргон не имеют стабильных магнитных изотопов, что делает их привлекательными для изучения изолированных молекул методом ЭПР.
 Криптон и ксенон обладают большей поляризуемостью, однако такие матрицы устойчивы при более высоких температурах: до приблизительно 50~K и 80~K, соответственно.
 Криптон содержит 11,48\% магнитных ядер $^{83}$Kr, поэтому суперсверхтонкая структура в ЭПР-спектрах в данной матрице не разрешена. Ксенон содержит значительное
количество изотопов $^{129}$Xe и $^{131}$Xe, за счёт этого в спектрах ЭПР появляется дополнительная сложная подструктура, затрудняющая их интерпретацию. В некоторых случаях применяют матрицу $^{136}$Xe, не содержащую магнитных ядер, для 
получения хорошо разрешённых спектров ЭПР \cite{Feldman2012_}.
 Изменяя матричное вещество, можно моделировать различное взаимодействие изолированной молекулы с окружением и таким образом изучать влияние слабых взаимодействий, например, на процессы фотолиза и радиолиза.
 
В условиях матричной изоляции могут быть применены различные спектроскопические методы изучения структуры и свойств изолированных частиц. Одним из широко распространённых 
высокоинформативных методов является ИК-спектроскопия \cite{Feldman2014}.
 При интерпретации спектроскопических данных необходимо учитывать различные эффекты, обусловленные матрицей.
Так, в спектрах частиц, изолированных в матрице, полосы поглощения имеют гораздо меньшую ширину по сравнению
со спектрами в твёрдой, жидкой или газовой фазах за счёт отсутствия сильных межмолекулярных взаимодействий. Частым явлением является <<матричный сдвиг>> ---
изменение частоты максимума полосы поглощения частицы в матрице по сравнению со спектром в газовой фазе, возникающее за счёт возмущения матрицей колебательной структуры
частицы. В матрице неона эти сдвиги минимальны, тогда как в более поляризуемых матрицах (в особенности, в ксеноне) могут достигать значительных величин.
Встречается расщепление полос поглощения за счёт разных типов положения частицы в матрице (<<сайтов>>).  Для небольших молекул, способных вращаться в матрице, может 
наблюдаться вращательная подструктура. 

Для генерации активных частиц при использовании матричной изоляции используют различные методы. Существует два основных подхода: генерация активных частиц в 
газовой фазе с последующим осаждением их вместе с матрицей и генерация непосредственно в осаждённой матрице. Для первого подхода применяют высокотемпературное испарение,
лазерную абляцию,
пиролиз,
газовый разряд,
фотолиз и радиолиз в газовой фазе. Для второго~--- фотолиз и радиолиз в матрице \cite{MI}.
Последние указанные два метода принципиально отличаются по механизму взаимодействия с веществом.
При фотолизе в матрице излучение поглощается селективно молекулами исследуемого вещества, матрица служит резервуаром для рассеивания энергии и стабилизирующей средой для продуктов, благодаря этому начальные этапы фотолиза в условиях матричной изоляции похожи на таковые в газовой фазе.  
В случае радиолиза основная часть энергии поглощается матрицей. Эффективность радиационно-индуцированного разложения вещества в матрице зависит от эффективности передачи энергии с матрицы на молекулы исследуемого вещества. Типичные значения радиационно-химических выходов разложения молекул в матрицах составляют несколько единиц на 100 эВ. Приведём несколько значений радиационно-химических выходов 
разложения различных веществ в матрице аргона: HCN~--- 3.8~молек./100~эВ, ацетонитрил~--- 5~молек./100~эВ \cite{Kameneve_diss}, бутан~--- 2.9~молек./100~эВ, гептан~--- 1.8~молек./100~эВ \cite{Feldman1999_}. Типичные процессы, происходящие при радиолизе органических веществ в условиях матричной изоляции представлены уравнениями \ref{eq1}--\ref{eq7} (где Ng~--- атом благородного газа, M~--- молекула изолированного вещества, P~--- продукты радиолиза) \cite{Feldman2014, Feldman1999_, Feldman1997}. 

\begin{equation}\label{eq1}
\mathrm{ 
Ng \leadsto Ng^{+\bullet}, e^-, Ng^*}
\end{equation}
\begin{equation}\label{eq2}
\mathrm{
Ng^{+\bullet} + M \to (M^{+\bullet})^* + Ng}
\end{equation}
\begin{equation}\label{eq3}
\mathrm{
Ng^* + M \to M^* + Ng}
\end{equation}
\begin{equation}\label{eq4}
\mathrm{
(M^{+\bullet})^* \to M^{+\bullet}}
\end{equation}
\begin{equation}\label{eq5}
\mathrm{
(M^{+\bullet})^* \to P}
\end{equation}
\begin{equation}\label{eq6}
\mathrm{
M^{+\bullet} + e^- \to M^{**}}
\end{equation}
\begin{equation}\label{eq7}
\mathrm{
M^* \;(M^{**})\to P}
\end{equation}

При взаимодействии излучения с матрицей образуются катион-радикалы матричных атомов, экситоны и электроны (уравнение~\ref{eq1}).
Затем происходит передача заряда и возбуждения с матрицы на молекулы изолированного вещества (уравнения~\ref{eq2}, \ref{eq3}). При передаче <<дырки>> с матрицы на изолированную молекулу образуется катион-радикал в возбуждённом состоянии из-за разности потенциалов ионизации атомов матрицы и молекул рассматриваемого вещества. Это  верно для многих органических веществ (часто потенциал ионизации около 10~эВ), так как инертные газы имеют высокие потенциалы ионизации (21.56~эВ для Ne, 15.75~эВ для Ar, 14.0~эВ для Kr, 12.13~эВ для Xe \cite{xray}). Полученный возбуждённый КР может релаксировать в основное состояние (уравнение~\ref{eq4}) или приводить к образованию продуктов (уравнение~\ref{eq5}). Кроме того, КР может рекомбинировать с электронами с образованием возбуждённой молекулы (уравнение~\ref{eq6}).  Из возбуждённых состояний, в свою очередь, могут образовываться продукты
(уравнение~\ref{eq7}). 

Катион-радикалы, образующиеся при передаче энергии с матрицы на изолированную молекулу, обладают избыточной энергией, близкой к разности потенциалов ионизации. Потому в матрицах с более высокими потенциалами ионизации КР могут не стабилизироваться, а сразу же претерпевать фрагментацию. При этом КР некоторых алканов могут быть стабилизированы в матрице ксенона \cite{Feldman1999_}. Однако поляризуемость матрицы влияет на стабилизацию КР противоположным образом. КР могут депротонироваться при наличии акцептора протонов. Способность принимать протон для инертных газов коррелирует с поляризуемостью, то есть увеличивается от неона к ксенону \cite{NIST}. В матрицах аргона, криптона и ксенона зафиксированы стабилизированные протоны типа Ng$_2$H$^+$, возникающие, по-видимому, в результате реакции \ref{eq8} \cite{Bondybey1972, Kunttu1992}. Благодаря этой реакции, КР, обладающие значительной кислотностью (вода, метанол и др.) стабилизируются только в матрице неона \cite{Knight1992, Knight1995}.

\begin{equation}\label{eq8}
\mathrm{
H^+ + 2Ng \to Ng_2H^+}
\end{equation}

 Соединения типа Ng$_2$H$^+$ имеют характерные полосы в ИК-спектрах:  903.3 и 1139.6~см$^{-1}$ для Ar$_2$H$^+$ \cite{Bondybey1972};  852.5, 1007.7 и 1160.4~см$^{-1}$ для Kr$_2$H$^+$ \cite{Bondybey1972};  730.6, 842.7 и 953.4~см$^{-1}$ для Xe$_2$H$^+$ \cite{Kunttu1992}. Аналогичные соединения образуются в случае стабилизации дейтрона. В ИК-спектре их можно наблюдать по полосам поглощения при 644~см$^{-1}$ для Ar$_2$H$^+$ \cite{Bondybey1972}; 606~см$^{-1}$ для  Kr$_2$D$^+$ \cite{Bondybey1972}; 516.7~см$^{-1}$ для  Xe$_2$D$^+$ \cite{Kunttu1992}. Соединения типа Ng$_2$H$^+$ неустойчивы и даже при температуре 7~K гибнут со временем в темноте, а также при фотолизе видимым светом \cite{Bondybey1972}.


\subsection{Радиолиз бензола}
\label{radiolysis}
Радиолизу бензола посвящено множество работ. Многие из них направлены на установление конечных продуктов разложения бензола, но не на установление 
детального механизма. Кроме анализа этих работ, в данном разделе приведены имеющиеся данные о промежуточных частицах, образующихся при радиолизе бензола.
\subsubsection{Конечные продукты радиолиза бензола}
\label{products}
Большое количество работ, направленных на идентификацию конечных продуктов радиолиза бензола, было опубликовано в 1960--1980~гг. 

Радиолиз бензола
в газовой фазе изучен слабо. Существует всего несколько работ, в которых рассмотрен этот процесс. 
Так, показано, что при радиолизе бензола в газообразном состоянии радиационно-химический выход разложения составляет 
4--6~молекул/100~эВ \cite{burns1969, walzbach1968}. В работе \cite{burns1969} исследовали радиолиз бензола при температурах 260--390$^\circ$С. 
При помощи газовой хроматографии наблюдали образование водорода, метана, ацетилена, этана, этилена, бифенила и полимерных продуктов. Были установлены зависимости радиационно-химических выходов перечисленных продуктов
от плотности и температуры (для всех продуктов, кроме ацетилена, выходы увеличиваются с ростом температуры и уменьшаются с увеличением плотности). Сильнее всего с ростом температуры менялся выход водорода (от 0.1 до 3~молекул/100~эВ).
Авторы полагают, что при повышении плотности выход разложения бензола снижался из-за конкуренции между дезактивацией возбуждённых состояний и их превращением в продукты.

При радиолизе бензола в конденсированном состоянии наблюдаются низкие радиационно-химические выходы газообразных продуктов. Так, при облучении $\gamma$-лучами или электронами
 жидкого бензола выход водорода составляет около 0.04~молекул/100~эВ, выход ацетилена~--- 0.02~молекул/100~эВ \cite{Chapiro1977, Cherniak1964}. 
 При радиолизе твёрдого бензола при 160~K радиационно-химические выходы ещё ниже: $G$(H$_2$)~=~0.0085~молекул/100~эВ,  $G$(CH$_4$)~=~0.008~молекул/100~эВ, $G$(C$_2$H$_2$)~=~0.0016~молекул/100~эВ \cite{Solid}.

 При радиолизе жидкого дейтерированного бензола образуется молекулярный дейтерий, его радиационно-химический выход меньше, чем выход молекулярного водорода 
(0.017~молекул/100~эВ). 
Выход дейтерированного ацетилена не сильно отличается от его недейтерированного аналога (0.0133~молекул/100~эВ). При радиолизе смеси C$_6$H$_6$--C$_6$D$_6$
(1:1) образуется водород с соотношением изотопологов H$_2$:HD:D$_2 = 52.1:33.1:14.8$ \cite{Gordon1952}.
 
 Выход тяжёлых продуктов при радиолизе бензола на несколько порядков выше. Основным продуктом является так называемый 
 <<полимер>> --- смесь веществ, полученных объединением нескольких молекул бензола. Радиационно-химический выход превращения бензола в <<полимер>> составляет 0.75~молекул/100~эВ \cite{Patrick1954}. 
 Отношение содержания углерода и водорода варьируется в <<полимере>> от 1.0 до 1.6, 
 средняя молекулярная масса растёт с поглощённой дозой и  может достигать 430~г/моль при дозе около 10$^7$~Гр \cite{Patrick1954}. 
 При помощи хроматографии показано, что при радиолизе жидкого бензола образуются бифенил, циклогексадиены, фенилциклогексадиены, бициклогексадиены, а так же различные терфенилы \cite{Cherniak1964, Zimmerli1969}.
 
В работе группы Страццуллы \cite{Strazzulla1991} чистый замороженный бензол облучали ионами гелия с энергией 3 кэВ.
По ИК-спектру облучённых образцов сделаны предположения о составе продуктов радиолиза. Широкие полосы молекулярной матрицы не дали возможности для точного отнесения.
Однако авторы называют одними из основных продуктов ацетилен и монозамещённый ацетилен, кроме того наблюдают полосы, относимые к С=С и C--H колебаниям в замещённом бензольном кольце, C--H колебаниям в алифатических фрагментах.
Сделано предположение о существовании продукта со структурой HC$\equiv$C\nobreakdash--CH$_2$\nobreakdash--C$\equiv$C\nobreakdash--C$_6$H$_5$.

 Р. Х. Шулер и Дж. А. ЛаВерне проводили систематические исследования радиолиза бензола тяжёлыми ионами. 
Изучен радиационно-химический выход водорода при бомбардировке протонами, дейтронами, ионами гелия \cite{Schuler1965},  $^7$Li \cite{LaVerne1982}, $^9$Be, $^{11}$B, $^{12}$C \cite{LaVerne1984}.
Установлено, что даже при малых энергиях тяжёлых частиц выход водорода значительно превышает выход при облучении электронами или $\gamma$-лучами.
Показан резкий рост радиационно-химического выхода водорода при увеличении ЛПЭ.

В работе \cite{LaVerne2002} проведено сравнение радиолиза жидкого бензола $\gamma$\nobreakdash-лучами и тяжёлыми ионами. При помощи газовой хроматографии с масс-спектрометрическим детектированием были определены выходы продуктов. Показано, что несмотря на то, что
 при $\gamma$\nobreakdash-радиолизе почти все возбуждённые состояния релаксируют до основного электронного состояния, при облучении тяжёлыми ионами реакции возбуждённых состояний приводят к значительным выходам продуктов.
 Основными зафиксированными продуктами названы бифенил, молекулярный водород и фенильный радикал. Последний регистрировали в экспериментах с добавками иода (по образованию иодбензола). 
  Показано, что фенильный радикал может образовывать с бензолом относительно долгоживущий аддукт, который затем ведёт к образованию <<полимеров>>.
 Выход бифенила не зависел от типа облучения и составлял 0.075~молекул/100~эВ. Кроме того, на основании сходства зависимостей флюоресценции и выхода H$_2$ от ЛПЭ авторы делают вывод о том, что предшественником молекулярного водорода является
 синглетное возбуждённое состояние.

 В 2005 году опубликована работа \cite{Ruiterkamp2005}, в которой было проведено сравнение УФ-фотолиза и бомбардировки протонами бензола. Кроме облучения чистого бензола 
были проведены исследования поведения молекул бензола, изолированных в матрице твёрдого аргона, а также в модельных кислородсодержащих астрохимических льдах.
Идентификацию продуктов и оценку эффективности разложения бензола проводили при помощи ИК-спектроскопии. Точное определение продуктов радиолиза и фотолиза не было основной целью работы,
однако авторы полагают, что основными продуктами в случае матрично-изолированного бензола являются продукты распада: ацетилен и метилацетилен. 
Кроме того, часть полос поглощения авторы относят к ассоциатам ацетилена и предположительно к комплексам ацетилен-метилацетилен. 
В спектре облучённого бензола в аргоновой матрице появляются значительные количества CO$_2$ и CO. Авторы объясняют это загрязнением, появляющимся в процессе долгого 
эксперимента. Кроме того, в спектре имеется большое количество не отнесённых полос.
 Сделан вывод, что передача энергии происходит намного эффективнее при радиолизе, чем при фотолизе: бензол разлагается примерно в 300 раз эффективнее при радиолизе, чем при фотолизе в расчёте на один
 поглощённый протон или фотон. 

 Таким образом, состав конечных продуктов радиолиза бензола изучен хорошо. Среди основных газообразных продуктов названы водород и ацетилен. Кроме газообразных продуктов образуется так называемый <<полимер>>. Показано, что радиационно-химические выходы газообразных продуктов низки. Однако, установлено, что в отличие от $\gamma$\nobreakdash-радиолиза и облучения электронами,  использование тяжёлых ионов приводит к значительным выходам продуктов. Показано, что радиационно-химические выходы продуктов, полученных при радиолизе бензола-$d_6$ ниже, чем соответствующих продуктов при радиолизе недейтерированного бензола.
 
 \subsubsection{Промежуточные частицы при радиолизе бензола}
 \label{intermediates}
 Перейдём к рассмотрению промежуточных частиц, образующихся в процессе радиолиза бензола. 
 Так, радиационно-химический выход ионных пар в жидком бензоле, оценённый при помощи метода растягивающего поля, составляет по различным данным от 0.052 до 0.081~и.п./100~эВ \cite{Schmidt1968, Schmidt1970, Shinsaka1974}.
 Рассмотрим далее более подробно образование возбуждённых состояний молекул бензола и радикалов в процессе радиолиза.
 
 \paragraph{Возбуждённые состояния при радиолизе бензола\\}
 \label{ex}
 
Были предприняты попытки оценить радиационно-химические выходы возбуждённых состояний при радиолизе жидкого бензола различными способами. 
Так, оценку радиационно-химического выхода триплетных возбуждённых состояний проводили 
с использованием {\it цис-транс-}изомеризации алкенов. Считалось, что изомеризация происходит при передаче возбуждения с бензола на алкен. 
Были получены значения выходов триплентно-возбуждённых молекул бензола 4.0--4.7~молекул/100~эВ \cite{Golub1966, Cundall1970}. 
Однако эти результаты стоит подвергнуть сомнению, так как было показано, что акцептор понижает выход триплетных состояний \cite{Hentz1969}, а 
изомеризация {\it цис}-бутена-2 является цепным процессом (выход около 4$\cdot10^3$~молекул/100~эВ) \cite{Harata1977}.

Другой подход к определению выходов возбуждения~--- метод импульсного радиолиза. Флуоресценцию облучённого бензола с максимумами при длинах волн  
279 и 285 нм относят к синглетному возбуждённому состоянию мономера бензола, а с максимумом при длине волны 320~нм~--- к 
эксимерному возбуждённому состоянию \cite{Horrocks1970, West1970}. В работе \cite{Thomas1969} с помощью наносекундного радиолиза бензола 
с добавками нафталина и антрацена радиационно-химический выход возбуждённого состояния S$_1$ бензола оценён как 1.6--1.7~молекул/100~эВ.
 Позднее с помощью пикосекундного импульсного радиолиза было изучено 
образование возбуждённых состояний в бензоле \cite{beck1972}. Показано, что синглетное возбуждённое состояние образуется за время 
меньшее 10~пс и  имеет время жизни около 20~нс. 
Зафиксировано образование эксимерного возбуждённого состояния бензола, характерное время его формирования оценено как 7~пс.

Из экспериментальных данных по радиолизу бензола с добавками акцепторов (COS и N$_2$O) был оценён радиационно-химический выход возбуждённых состояний S$_1$, образующихся не при
 рекомбинации электронов с катион-радикалами, а путём прямого возбуждения. Он составляет не более 0.2~молекул/100~эВ \cite{Sato1972}.
 
 В работе \cite{HorrocksDL1970} радиационно-химический выход S$_1$ состояний бензола, образующихся при прямом возбуждении и путём внутримолекулярной конверсии, 
 оценён с использованием добавок с низколежащими уровнями возбуждённых состояний, выход составил 0.4~молекул/100~эВ.

В работе \cite{Baxendale1972} был оценён выход триплетно-возбуждённых молекул бензола при помощи техники импульсного радиолиза. 
Авторы провели ряд экспериментов с акцепторами (бифенил, антрацен, нафталин). Зависимости выходов триплетных состояний от концентрации акцептора 
являются линейными в координатах 1/$G$(T)--1/$C$ (кинетика Штерна-Вольмера).
Начальный выход триплетных возбуждённых состояний составил 4.2~молекул/100~эВ. Период полураспада оценён как 20 нс.

Существуют другие оценки радиационно-химических выходов триплетных и синглетных возбуждённых состояний. Так, в работе \cite{Cundall1968} по фосфоресценции 
биацетила при микросекундном 
импульсном радиолизе получены оценки: выход триплетов --- 1.24~молекул/100~эВ, выход синглетов~--- 1.43~молекул/100~эВ.
В работе \cite{Cooper1968} при помощи наносекундного импульсного радиолиза бензола с добавкой пиперилина выход триплетов 
оценён как 1.85~молекул/100~эВ,
выход синглетов~--- 1.62~молекул/100~эВ.

В работе \cite{Land1968} исследовали зависимость радиационно-химического выхода триплено-возбуждённых состояний от концентрации добавки (акцептора: нафталина, антрацена или бифенила) в бензоле.
Авторы делают вывод о том, что при концентрациях добавки ниже 0.1 М, в основном, происходит перенос энергии с возбуждённых молекул бензола на молекулу акцептора,
однако при более высоких концентрациях возможно формирование дополнительного количества триплетно-возбуждённых молекул акцептора за счёт нейтрализации ионов добавки.

Несмотря на неоднозначность данных о выходах возбуждённых состояний при радиолизе бензола, автор \cite{z1} называет наиболее 
вероятными значения радиационно-химических 
выходов триплетно-возбуждённых состояний бензола 4.2~молекул/100~эВ, синглетно-возбуждённых~--- 1.5--1.6~молекул/100~эВ.

В более поздней работе \cite{Okamoto2003} с помощью субпикосикундного радиолиза стало возможным напрямую наблюдать рекомбинацию димерных катион-радикалов и электронов.
Среднее время рекомбинации оценено как 1.2~пс.

Таким образом, образование возбуждённых состояний бензола, возникающих при радиолизе, изучено различными методами, в том числе при помощи импульсного радиолиза. Данные о радиационно-химических выходах возбуждённых состояний, полученные в различных работах, отличаются. Наиболее вероятными значениями выходов синглетных и триплетных возбуждённых состояний являются   1.5--1.6~молекул/100~эВ и 4.2~молекул/100~эВ, соответственно.

\paragraph{Радикалы при радиолизе бензола\\}
\label{rad}

Существует ряд работ, направленных на идентификацию радикалов, образующихся при радиолизе бензола и определении их выходов. 
Эксперименты с использованием  акцепторов приводят к значениям суммарного выхода радикалов 0.7--0.9~радикалов/100~эВ~\cite{Cherniak1964, Weber1955}.
В качестве акцепторов использовались иод и дифенилпикрилгидразил. Однако данные могут быть не точны, так как указанные молекулы 
являются акцепторами не только радикалов, но и возбуждённых состояний. Кроме того, образующиеся радикалы способны вступать 
в реакции с бензольным кольцом.

Состав и радиационно-химические выходы радикалов в жидком бензоле исследовали при помощи метода спиновых ловушек~\cite{Sargent1977}. 
При использовании в качестве ловушки 2,4,6-три-{\it трет}-бутилнитрозобензола зафиксированы аддукты ловушки с фенильным и 
циклогексадиенильным радикалами~\cite{ST1}. Суммарный выход радикалов оценён как 0.04--0.06~радикалов/100~эВ.
Однако позднее было показано, что наблюдаемый спектр связан не с наличием в системе C$_6$H$_7^\bullet$ радикалов, а с продуктами радиолиза спиновой ловушки~\cite{ST2}.
При использовании в качестве ловушек нитрозодурола и фенилбутилнитрона выход фенильных радикалов составляет 0.05--0.06~радикалов/100~эВ~\cite{ST1, ST3}. 

При облучении бензола в твёрдой фазе был зафиксирован радикал C$_6$H$_7^\bullet$.
В облучённом поликристаллическом бензоле при 4.2~K наблюдали возбуждаемую лазером флуоресценцию, относимую к радикалу C$_6$H$_7^\bullet$ 
(к C$_6$D$_7^\bullet$ при радиолизе бензола-$d_6$)~\cite{Sheng1978}. При помощи метода ЭПР в облучённом бензоле при 77~K изучено образование радикальных пар, 
сделано предположении об образовании радикальных пар C$_6$H$_7^\bullet$--C$_6$H$_5^\bullet$~\cite{Matsuyama1978a, Matsuyama1978}.
Кроме того, зафиксировано образование радикала C$_{12}$H$_{11}^\bullet$ (продукта присоединения фенильного радикала к бензолу)~\cite{Fessenden1963}. 

 Группой под руководством В.~И.~Фельдмана был выполнен ряд работ по радиолизу бензола, изолированного в матрицах твёрдых благородных газов. Так, было показано, что в 
 матрице ксенона основным каналом радиационно-индуцированных превращений бензола является распад на атом водорода и фенильный радикал~\cite{Feldman2007}. Образование продуктов фиксировали 
 при помощи ИК-  и ЭПР-спектроскопии. Появляющиеся после облучения полосы в ИК-спектрах с волновыми числами 3056, 1437,
1429, 1023, 703, 655 см$^{-1}$ относят к фенильному радикалу. В спектрах ЭПР наблюдался сигнал изолированных атомов водорода, представляющий собой дублет с расщеплением около 50.6 мТл, имеющий подструктуру, обусловленную взаимодействием
неспаренного электрона с магнитными атомами ксеноновой матрицы. Кроме того, в спектрах ЭПР присутствовал сигнал фенильных радикалов. Константы СТВ составляют для {\it орто}-H $a_{xx}$~= 2.19~мТл,
 $a_{yy}$~= 1.54~мТл, $a_{zz}$~= 1.49~мТл, для двух {\it мета}-H $a_{xx}$~= 0.66~мТл, $a_{yy}$~= 0.61~мТл, $a_{zz}$~= 0.50~мТл и для {\it пара}-H $a_{xx}$~= 0.20~мТл, $a_{yy}$~= 0.25~мТл,
 $a_{zz}$~= 0.12~мТл \cite{Kasai1969}. Кроме того, было показано, что при отжиге при 45 K образуется радикал C$_6$H$_7^\bullet$. Его наличие подтверждается появлением в ЭПР-спектре дополнительного сигнала,
представляющего собой триплет квартетов с константами 4.85 и 1.05~мТл~\cite{Pshezhetskii}. В ИК-спектре после отжига появляются полосы поглощения
 с  волновыми числами 2768, 1387, 1287, 958, 908, 620, 618 и 546~см$^{-1}$, которые авторы \cite{Feldman2007} отнесли к циклогексадиенильному радикалу.
 
В работах \cite{Feldman1999, Feldman2000} при помощи метода ЭПР изучено образование катион-радикала бензола. В работе \cite{Feldman1999} 
показано, что полученный в матрице твёрдого аргона с добавкой CFCl$_3$ катион-радикал бензола стабилизируется в $^2$B$_{1g}$ состоянии. Позднее 
авторы продемонстрировали сильное влияние матрицы на стабилизацию катион радикала~\cite{Feldman2000}. В отличие от матрицы аргона (спектр КР: квинтет с константой СТВ 0.64~мТ~(4H)), в  
матрицах криптона и ксенона два состояния катион-радикала переходят друг в друга даже при низких температурах эксперимента (10--12~K). Спектр ЭПР облучённого образца в этом случае представляет собой неразрешённый синглет.
Авторы предполагают, что
это связано с большим размером ловушек за счёт большего радиуса атомов и более лёгкой деформаций кристаллической решётки матрицы. Кроме данных о катион-радикале бензола в матрицах благородных газов, 
были получены ЭПР-спектры C$_6$H$_6^{+\bullet}$ в замороженном SF$_6$ (при 12~K неразрешённый синглет, при 93~K константа СТВ 0.45~мТл~(6H)) и на цеолите H-ZCM-5 (константа СТВ 0.45~мТл~(6H)).

Таким образом, данные о радиационно-химических выходах радикалов при радиолизе бензола, полученные разными методами, значительно различаются. Кроме того, при радиолизе бензола показано образование  фенильного радикала, в том числе в условиях матричной изоляции. Известен ЭПР-спектр КР бензола, в том числе в матрицах твёрдых благородных газов, однако до сих пор нет надёжных данных о его колебательном спектре.

\subsection{Фотолиз бензола}
\label{photolysis}
Фотолиз бензола изучен широко, опубликовано множество работ, посвящённых этой теме (см., например, обзоры~\cite{Bryce-Smith1976, Bryce-Smith1977}). Информация о продуктах фотолиза и механизмах их образования могут быть полезны для интерпретации экспериментальных данных по радиолизу бензола. 

Показано, что в жидкой фазе основными продуктами являются изомеры бензола,
при этом состав продуктов зависит от длины волны излучения. Так, возбуждение в состояние S$_1$ (фотолиз излучением с длиной волны 254~нм) приводит к образованию 
бензвалена (рисунок~\ref{b}) и фульвена (рисунок~\ref{f}) \cite{Wilzbach1967, Wilzbach1968}. При облучении на длине волны 203~нм (достаточно для возбуждения в состояние~S$_2$) 
кроме указанных изомеров образуется бензол Дьюара (рисунок~\ref{bD}) \cite{Ward1968, Bryce-Smith1971}. \\

\begin{figure}[H]  
\vspace{-4ex} \centering \subfigure[]{
\includegraphics[width=0.1\linewidth]{b.eps} \label{b}}  
\hspace{4ex}
\subfigure[]{
\includegraphics[width=0.1\linewidth]{f.eps} \label{f}}
\hspace{4ex}
\subfigure[]{
\includegraphics[width=0.1\linewidth]{bD.eps} \label{bD}}  
\caption{\subref{b} бензвален; \subref{f} фульвен; \subref{bD} бензол Дьюара} \label{iso}
\end{figure}

При фотолизе бензола в газовой фазе излучением с длиной волны 193~нм образуются фульвен, {\it цис}- и {\it транс}\nobreakdash-гексадиен\nobreakdash-1,3\nobreakdash-ин\nobreakdash-5, небольшие количества фенильного радикала, однако не 
образуется бензвален и бензол Дьюара \cite{Ward1968a, Kaplan1968a, Tsai2000}. Было сделано предположение, что бензвален и бензол Дьюара формируются из колебательно-возбуждённого основного электронного состояния 
молекулы бензола (так называемого <<горячего бензола>>) \cite{Ward1968a, Nakashima1989, Yatsuhashi2001}. Таким образом, их образование возможно, если электронно-возбуждённая молекула бензола претерпевает безызлучательную конверсию в состояние S$_0$.
Данный процесс эффективно протекает в жидкости из-за межмолекулярных столкновений. В газовой фазе этот процесс неэффективен, потому образуется только термически стабильный фульвен
и продукты фотодиссоциации (таких как, ацетилен, пропилен, гексадиен-1,3-ин-5 и др.) \cite{Bryce-Smith1976}. Известно, что {\it транс}\nobreakdash-гексадиен\nobreakdash-1,3\nobreakdash-ин\nobreakdash-5 под действием излучения с длиной волны 254 нм переходит в {\it цис}-изомер. Гексадиен\nobreakdash-1,3\nobreakdash-ин\nobreakdash-5 затем под действием такого же излучения изомеризуется в бензол и фульвен \cite{Bryce-Smith1976}.


Фотолиз бензола в матрицах изучен мало. Так, в работе \cite{Johnstone1991} описан УФ-фотолиз бензола в матрице твёрдого аргона. При использовании излучения с длиной волны 254~нм 
по данным ИК-спектроскопии первичными продуктами являются фульвен, бензвален и бензол Дьюара (накапливаются линейно с увеличением времени фотолиза). Эксперимент был повторён в матрице ксенона, однако продуктов фотолиза не наблюдалось.
Авторами сделано предположение, что состояние T$_1$ не является предшественником бензола Дьюара, так как в матрице ксенона из-за эффекта тяжёлого
атома выход данного состояния должен быть высоким, однако образования бензола Дьюара не наблюдалось. Для его образования в матрице аргона 
предложен механизм, включающий смешение S$_1$ и S$_2$ состояний, вызванное матрицей. Показано, что при фотолизе бензола в матрице аргона кривая накопления бензвалена со временем выходит на стационарное значение концентрации.
Сделано предположение, что полученный бензвален разлагается при фотолизе с образованием бензола и фульвена. Этот вывод согласуется с ранее полученными данными по фотолизу бензвалена в газовой фазе \cite{Kaplan1968, Harman1981}.  

В работе \cite{Laboy1993} сообщается, что фотолиз осаждённых смесей бензол/аргон и бензол-$d_6$/аргон (в соотношениях от 1:250 до 1:2000) лазером с длиной волны излучения 193~нм не приводит к образованию продуктов по данным ИК-спектроскопии.
Авторы применили фотолиз газовой смеси в процессе осаждения на холодную подложку. При такой постановке эксперимента наблюдались продукты изомеризации и фрагментации бензола.
Показано наличие таких изомеров, как фульвен, бензвален, бензол Дьюара и призман. Среди продуктов фрагментации названы бензин, пропин, 1,3-бутадиин, 
гексатриин, ацетилен и этилен. Кроме того, малоинтенсивная полоса с волновым числом 708~см$^{-1}$ отнесена к фенильному радикалу. Показано, что отношение выходов 
изомеризации и фрагментации бензола зависит от его концентрации в матрице: роль фрагментации значительна при низких концентрациях бензола, изомеризация преобладает в случае больших концентраций.  
Авторы объясняют это различной эффективностью передачи возбуждения в зависимости от концентрации бензола. 
Так, при малой концентрации релаксация неэффективна и быстро протекает
фрагментация, а при большой концентрации эффективность релаксации повышается за счёт столкновений возбуждённых молекул бензола с невозбуждёнными в процессе осаждения. 
Столкновения возбуждённых 
молекул бензола с атомами аргона, конечно, происходят во всех случаях, однако передача энергии не происходит из-за более низкого потенциала 
возбуждения бензола, чем аргона. Кроме того, авторы отмечают отсутствие бензола Дьюара
в экспериментах с низкой концентрацией бензола. Основываясь на этом, они полагают, что в образовании данного продукта принимают участие две молекулы бензола.
Описанная концентрационная зависимость полос была применена авторами к фотолизу дейтеробензола. Полосы продуктов были разделены на две группы, относимые к продуктам изомеризации и фрагментации.
Авторы по имеющимся на тот момент литературным данным относят некоторые серии полос к CD$_3$C$\equiv$CD, DC$\equiv$C--C$\equiv$CD, C$_6$D$_4$ и CD$_2$=CD$_2$. 

В 2015 году опубликована работа \cite{Toh2015}, в которой изучали фотолиз бензола в матрице твёрдого {\it пара}-водорода при помощи ИК-спектроскопии. Использовали излучение с длинами волн 193.0 и 253.7~нм. Основными продуктами фотолиза названы изомеры бензола: 
фульвен, бензол Дьюара и бензвален. Авторы называют первичными продуктами при фотолизе излучением с длиной волны 193.0~нм все три изомера на основании линейных зависимостей их концентраций от времени фотолиза. 
При использовании излучения с длиной волны 253.7~нм фульвен, в отличие от двух других изомеров, не появляется при малых временах фотолиза. Авторы 
делают вывод о том, что фульвен образуется из бензвалена, что согласуется с  более ранними исследованиями, в которых было показано, что фотолиз бензвалена излучением 
с длиной волны 253.7~нм приводит к образованию смеси бензола и фульвена в отношении 3:1 \cite{Kaplan1968, Harman1981}. 
Теоретически предсказана возможность образования и фульвена, и бензвалена из состояния S$_1$ через один и тот же предшественник (<<префульвен>>) \cite{Dreyer1996, Jano1968}, однако эксперименты авторов \cite{Toh2015} её не подтвердили.
Кроме того, в процессе фотолиза бензола в матрице {\it пара}-водорода зафиксирован циклогексадиенильный радикал, который претерпевает раскрытие цикла с образованием гексатриена\nobreakdash-1,3,5. 
Были зафиксированы {\it о}-бензин и гексадиен\nobreakdash-1,3\nobreakdash-ин\nobreakdash-5, образующиеся в малых концентрациях при фотолизе бензола. 
 
 Таким образом, в большом количестве работ по фотолизу бензола в газовой и жидкой фазах установлено образование продуктов фрагментации и изомеризации. Однако фотолизу бензола в условиях матричной изоляции посвящены лишь отдельные работы.
  
\vspace{1cm}
 Проведённый обзор литературы позволяет заключить, что бензол считается радиационно стойким; при этом мнение о его радиационной стойкости
 основано на крайне низких (по сравнению с алифатическими углеводородами) радиационно-химическими выходах газообразных продуктов при радиолизе в конденсированных фазах. Однако вопрос, обусловлена ли радиационная стойкость бензола его <<молекулярными>> свойствами или является следствием взаимодействия их молекул   с окружением, остаётся открытым. Для ответа на этот вопрос могут быть использованы эксперименты с использованием метода матричной изоляции, которая позволяет изучать свойства молекулы в инертном жёстком окружении. Кроме того, при использовании этого метода можно постадийно изучать механизмы радиолиза,
так как в условиях матричной изоляции высокореакционные интермедиаты могут быть стабилизированы и доступны для спектроскопического изучения.
   Однако работы, в которых обсуждалась возможная природа стойкости бензола на молекулярном уровне, единичны. Представления о начальных стадиях радиолиза бензола расплывчаты, данные о механизме радиолиза в бензола в матрицах отрывочны. Известно, что выходы продуктов при радиолизе бензола-$d_6$ ниже, чем при радиолизе недейтерированного аналога, однако влияние  дейтерирования на соотношение основных каналов радиолиза не установлено.
 
 Исходя из вышесказанного, в данной работе были поставлены следующие задачи:
 \begin{itemize}
\item определение состава радикальных и молекулярных продуктов радиолиза бензола в матрицах твёрдых благородных газов;
\item установление основных каналов радиационно-химических превращений молекул бензола в матрицах;
\item выявление влияния характеристик инертной матрицы на эффективность и механизм радиолиза бензола;
\item установление влияния изотопозамещения на основные каналы радиационно-химических превращений бензола в условиях матричной изоляции.
\end{itemize}
 
 }


 
 
 
 
 
 
 
 
 
 
 
 
 

 
 
 
 
 
 
 
 

