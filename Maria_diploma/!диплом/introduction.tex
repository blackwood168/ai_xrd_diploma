\newpage
\section*{Введение}
\addcontentsline{toc}{section}{Введение}
\sloppy{
Проблема молекулярной эволюции органического вещества в космическом пространстве находится в фокусе современного естествознания на протяжении нескольких десятилетий и привлекает постоянное внимание специалистов в области физики, химии и биологии. Один из наиболее широко обсуждаемых аспектов заявленной проблемы связан с механизмом образования полициклических ароматических углеводородов (ПАУ), обнаруженных в межзвёздных объектах различных типов.

В последние два десятилетия интерес к этой проблеме возрос в связи с успехами современной радиоастрономии и молекулярной астрофизики, позволившими надёжно зарегистрировать наличие достаточно большого числа относительно сложных молекул в различных космических объектах, а также планированием и осуществлением амбициозных космических миссий по исследованию дальнего космоса. Несмотря на очевидные успехи этих работ в последние годы, их выводы отчасти спекулятивны и нуждаются в экспериментальном и теоретическом обосновании.

Гипотеза об образовании ПАУ в межзвёздном пространстве была сформулирована на основе наблюдательных данных более 30 лет назад. С тех пор появилось большое число работ в этом направлении. Наличие ПАУ и других ароматических молекул (в основном, в ионизированном состоянии) в холодных межзвёздных средах не вызывает никаких сомнений, но в настоящее время остаётся открытым вопрос о механизмах, которые могут приводить к образованию столь сложных молекул ПАУ в межзвёздном пространстве.

Центральное место на пути построения сложных ароматических структур отводится бензолу, обнаруженному в диффузных молекулярных облаках. Реакции радикальных и ионных интермедиатов, возникающих из бензола (C$_6$H$_5^\bullet$, C$_6$H$_5^+$, C$_6$H$_7^+$), имеют ключевое значение для синтеза ПАУ во всех вариантах предложенных схем. Однако такие пути вовсе не очевидны с точки зрения традиционных представлений о фотохимии и радиационной химии бензола. Известно, что фотолиз бензола в конденсированных средах, в основном, приводит к образованию различных валентных изомеров, в которых не сохраняется ароматическое кольцо, а фенильных радикалов практически не образуется. Те же основные продукты, хотя и в другом соотношении, были обнаружены при УФ/ВУФ фотолизе матрично-изолированного бензола (в твёрдом пара-водороде наблюдался также радикал C$_6$H$_7^\bullet$ вследствие реакций атомов водорода из матрицы). Природа возбуждённых состояний, ответственных за реализацию различных каналов, остаётся дискуссионной, практически нет данных о реакциях высших триплетных состояний. С другой стороны, в радиационной химии имеется устойчивый стереотип о <<радиационной стойкости>> бензола на молекулярном уровне (очень низкий выход молекулярного водорода и фенильных радикалов). В действительности, однако, эти данные относятся не к молекулам бензола, а к жидкому или твёрдому бензолу, для которого характерна эффективная диссипация энергии вследствие образования эксимеров.

Данная работа посвящена моделированию важнейших элементарных стадий низкотемпературных радиационно-индуцированных процессов, приводящих в конечном итоге к образованию ПАУ. В качестве ключевой промежуточной структуры рассматривается молекула бензола – простейшая ароматическая молекула, обнаруженная в межзвёздном пространстве. Смысл использованного подхода состоит не в имитации условий и состава среды, а в направленном варьировании характеристик матриц (потенциал ионизации, поляризуемость, жёсткость) для моделирования каналов превращений, протекающих через различные ионные и возбуждённые состояния.

Такой подход позволяет, в принципе, установить общие закономерности ранних стадий радиационно-химических превращений органических молекул в жёстком окружении при криогенных температурах и сформулировать представления о механизмах влияния окружения на эти превращения, что имеет большое значение как для лабораторной астрохимии, так и для фундаментальной радиационной химии.

}