\newpage
\section{Результаты и выводы}
\begin{itemize}
\item Показано, что бензол и бензол-$d_6$ эффективно разлагаются в матрицах твёрдых благородных газов при температуре 6 K. При этом радиационно-химический выход разложения бензола в матрице аргона значительно выше, чем в матрицах криптона и ксенона (C$_6$H$_6$: Ar~--- 2.6~молек./100~эВ, Kr и Xe --- 0.4~молек./100~эВ; C$_6$D$_6$: Ar~--- 1.5~молек./100~эВ, Kr и Xe --- 0.3~молек./100~эВ).
\item Определён состав основных первичных продуктов радиолиза бензола и бензола-$d_6$ (фульвен и фенильный радикал) в матрицах благородных газов, предложена схема их образования. Состав продуктов принципиально отличается от состава продуктов, образующихся при фотолизе бензола.
\item Установлено, что дейтерирование не оказывает существенного влияния на соотношение основных каналов радиолиза бензола. Впервые получены ИК-спектроскопические характеристики дейтерированных изомеров бензола (фульвена-$d_6$, бензвалена-$d_6$, бензола Дьюара-$d_6$).
\item Показано, что матрица оказывает сильное влияние на соотношение каналов радиолиза изолированных молекул бензола: при переходе от матрицы аргона к матрице ксенона резко увеличивается относительный вклад канала радиационно-индуцированного распада бензола на атом водорода и фенильный радикал.
 \item Зафиксировано образование молекул с открытой цепью ({\itshape цис}- и {\itshape транс}-гексадиен\nobreakdash-1,3\nobreakdash-ина\nobreakdash-5) непосредственно. Предложены возможные механизмы их образования.
\end{itemize}

