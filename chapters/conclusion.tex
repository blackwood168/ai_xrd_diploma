%\section{Заключение}

%В данной работе представлен комплексный подход к решению проблемы фаз в рентгеновской кристаллографии с использованием методов глубокого обучения. Разработанное программное обеспечение для генерации синтетических рентгенодифракционных данных (\url{github.com/blackwood168/xrd_simulator}) создает надежную основу для обучения и тестирования моделей машинного обучения. Этот инструмент позволяет генерировать реалистичные дифракционные картины с учетом физических закономерностей, что критически важно для создания качественного обучающего набора данных.

%Для обеспечения воспроизводимости экспериментов и стандартизации процесса разработки был создан единый конвейер (\url{github.com/blackwood168/xrd_phase_ml}), который включает в себя все этапы работы с данными: от их подготовки до обучения моделей и оценки их эффективности. Это значительно упрощает процесс разработки и позволяет сравнивать различные архитектуры на единой основе.

%В рамках исследования были разработаны и протестированы три различные архитектуры нейронных сетей. В качестве базовой модели использовалась UNet, которая послужила отправной точкой для сравнения. FFT\_UNet, модифицированная версия UNet с встроенным преобразованием Фурье в слоях, была разработана с учетом специфики дифракционных данных. Особый интерес представляет XRD\_Transformer, архитектура на основе трансформера со специальным векторным представлением, который учитывает физические особенности задачи фаз. Все модели были обучены как на синтетических, так и на реальных кристаллических структурах, что позволило оценить их способность к обобщению.
%Проведенный анализ внутренней работы моделей с использованием методов интерпретируемого ИИ (GradCAM, карты внимания, анализ чувствительности) подтвердил, что модели действительно выявляют некоторые фундаментальные кристаллографические закономерности. 

%Текущие модели пока не достигают необходимого качества численного восстановления рентгеновских отражений для практического решения проблемы фаз. Это ограничение может быть связано с недостаточным учетом физических ограничений в архитектуре моделей и возможным несоответствием функции потерь физическим требованиям задачи. Для преодоления этих ограничений предлагается комплексный план изменений в подходе, включающий модификацию архитектуры моделей для лучшего учета физических ограничений, разработку специализированных функций потерь, учитывающих специфику проблемы фаз, и улучшение процесса обучения с использованием физически обоснованных ограничений

\section{Результаты и выводы}

\begin{enumerate}
\item Разработано программное обеспечение по генерации синтетических рентгенодифракционных данных, которое может быть использование для решения прикладных задач рентгеновской дифракции с помощью ИИ.%, а также автоматизированный конвейер, позволяющий проводить воспроизводимые эксперименты по обучению и тестированию моделей машинного обучения, повышающих разрешение рентгенодифракционных данных. %(\url{github.com/blackwood168/xrd_simulator})

\item Разработан единый автоматизированный конвейер, позволяющий проводить воспроизводимые численные эксперименты по обучению и тестированию моделей машинного обучения, повышающих разрешение рентгенодифракционных данных.


\item Лучшей разработанной моделью оказалась FFT\_UNet, предсказывающая ненормализованные амплитуды, фактор расходимости при уточнении структуры 6-бром-1Н-индол-2,3-диона составил 9.6\%.

%\item Разработаны и обучены на дифракционных данных синтетических и реальных структур UNet в качестве базовой модели, FFT\_UNet~---кастомный UNet с Фурье-преобразованием в слоях, XRD\_Transformer~--- модель на основе трансформер с векторным представлением, учитывающим физическую специфику задачи

\item Проведена интерпретация моделей, подтверждающая, что модели глубокого обучения выявляют кристаллографические закономерности и имеют потенциал решения прикладных задач рентгеновской дифракции.

\item У моделей машинного обучения пока не достает качества численного восстановления рентгеновских отражений для решения проблемы фаз в случае нецентросимметричных структур.

\end{enumerate}

