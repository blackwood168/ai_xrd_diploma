\section{Выводы}

\begin{itemize}
\item Разработано программное обеспечение по генерации синтетических рентгенодифракционных данных, которое может быть использование для решения прикладных задач рентгеновской дифракции с помощью ИИ (\url{github.com/blackwood168/xrd_simulator})

\item Разработан единый пайплайн (\url{github.com/blackwood168/xrd_phase_ml}), позволяющий проводить воспроизводимые эксперименты по обучению, тестированию и inference задачи предсказывания дальних отражений по ближним для решения проблемы фаз

\item Разработаны и обучены на синтетических и реальных структурах UNet в качестве baseline, FFT\_UNet --- кастомный UNet с Фурье-преобразованием в слоях, XRD\_Transformer --- модель на основе трансформер со специфичным эмбеддингом, вписывающимся в физику задачи

\item Проведен анализ моделей, подтверждающий, что модели глубокого обучения выявляют кристаллографические закономерности и имеют потенциал решения прикладных задач рентгеновской дифракции

\item У моделей не достает качества численного восстановления рентгеновских отражений для решения проблемы фаз, несмотря на улавливание рентгенодифракционных паттернов; приведено обоснование и предложен план изменения методологии для решения задачи

\end{itemize}

