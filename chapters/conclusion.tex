\section{Заключение}

В данной работе представлен комплексный подход к решению проблемы фаз в рентгеновской кристаллографии с использованием методов глубокого обучения. Разработанное программное обеспечение для генерации синтетических рентгенодифракционных данных (\url{github.com/blackwood168/xrd_simulator}) создает надежную основу для обучения и тестирования моделей машинного обучения. Этот инструмент позволяет генерировать реалистичные дифракционные картины с учетом физических закономерностей, что критически важно для создания качественного обучающего набора данных.

Для обеспечения воспроизводимости экспериментов и стандартизации процесса разработки был создан единый конвейер (\url{github.com/blackwood168/xrd_phase_ml}), который включает в себя все этапы работы с данными: от их подготовки до обучения моделей и оценки их эффективности. Это значительно упрощает процесс разработки и позволяет сравнивать различные архитектуры на единой основе.

В рамках исследования были разработаны и протестированы три различные архитектуры нейронных сетей. В качестве базовой модели использовалась UNet, которая послужила отправной точкой для сравнения. FFT\_UNet, модифицированная версия UNet с встроенным преобразованием Фурье в слоях, была разработана с учетом специфики дифракционных данных. Особый интерес представляет XRD\_Transformer, архитектура на основе трансформера со специальным векторным представлением, который учитывает физические особенности задачи фаз. Все модели были обучены как на синтетических, так и на реальных кристаллических структурах, что позволило оценить их способность к обобщению.
Проведенный анализ внутренней работы моделей с использованием методов интерпретируемого ИИ (GradCAM, карты внимания, анализ чувствительности) подтвердил, что модели действительно выявляют некоторые фундаментальные кристаллографические закономерности. 

Текущие модели пока не достигают необходимого качества численного восстановления рентгеновских отражений для практического решения проблемы фаз. Это ограничение может быть связано с недостаточным учетом физических ограничений в архитектуре моделей и возможным несоответствием функции потерь физическим требованиям задачи. Для преодоления этих ограничений предлагается комплексный план изменений в подходе, включающий модификацию архитектуры моделей для лучшего учета физических ограничений, разработку специализированных функций потерь, учитывающих специфику проблемы фаз, и улучшение процесса обучения с использованием физически обоснованных ограничений

\section{Выводы}

\begin{itemize}
\item Разработано программное обеспечение по генерации синтетических рентгенодифракционных данных, которое может быть использование для решения прикладных задач рентгеновской дифракции с помощью ИИ (\url{github.com/blackwood168/xrd_simulator})

\item Разработан единый пайплайн (\url{github.com/blackwood168/xrd_phase_ml}), позволяющий проводить воспроизводимые эксперименты по обучению, тестированию и inference задачи предсказывания дальних отражений по ближним для решения проблемы фаз

\item Разработаны и обучены на синтетических и реальных структурах UNet в качестве baseline, FFT\_UNet --- кастомный UNet с Фурье-преобразованием в слоях, XRD\_Transformer --- модель на основе трансформер со специфичным эмбеддингом, вписывающимся в физику задачи

\item Проведен анализ моделей, подтверждающий, что модели глубокого обучения выявляют кристаллографические закономерности и имеют потенциал решения прикладных задач рентгеновской дифракции

\item У моделей не достает качества численного восстановления рентгеновских отражений для решения проблемы фаз, несмотря на улавливание рентгенодифракционных паттернов; приведено обоснование и предложен план изменения методологии для решения задачи

\end{itemize}

