\section{Обзор литературы}

\subsection{Рентгеновская кристаллография}


Рентгеновская кристаллография является важнейшим инструментом определения трехмерных структур кристаллов. Этот метод позволяет получить бесценные сведения об атомных и молекулярных структурах кристаллов, что крайне важно для понимания свойств и функций материалов в различных областях, включая химию, биологию и материаловедение. Рентгеноструктурный анализ основан на упругом рассеянии монохроматического рентгеновского излучения на трехмерной регулярной решетке атомов твердого вещества, что приводит к интерференции рентгеновских лучей.

Дифракция в кристалле описывается как отражения от семейств параллельных кристаллографических плоскостей элементарной ячейки кристалла (рис. \ref{hkl}). Каждая точка дифракционной картины задается набором из индексов Миллера $h,k,l$. Любое отражение имеет математическое представление, известное как структурный фактор $F(h,k,l)$, зависящий от расположения и коэффициентов рассеяния всех
атомов в элементарной ячейке кристалла. Структура кристалла получается при расчете электронной плотности с помощью Фурье-преобразования: \[\rho(x, y, z) = V^*\sum_{h,k,l} F(h,k,l) exp[(-2\pi i (hx+ky+lz)],\] где $V^*$~-- объем элементарной обратной ячейки \cite{girolami_x-ray_2016}. 

\begin{figure}[H]
	\centering
	\includegraphics[width=0.8\textwidth]{figures/direct_reciprocal.png}\hfill
	\caption{Связь прямого и обратного пространства \cite{girolami_x-ray_2016}}
	\label{hkl}
\end{figure}




\subsection{Проблема фаз}

Дифракция в кристалле описывается как отражения от семейств параллельных кристаллографических плоскостей элементарной ячейки кристалла. Каждая точка дифракционной картины задается вектором из индексов Миллера $\overrightarrow{H} = (h,k,l)$, атомы --- вектором $\overrightarrow{r} = (x,y,z)$. Любое отражение имеет математическое представление, описывающее общую рассеянную от всех атомов элементарной ячейки волну, известное как структурный фактор $F(h,k,l)$:

\begin{equation}\label{eq1}
\mathrm{ 
F(h,k,l) = \sum\limits_{j=1}^N F_j (h,k,l) = \sum\limits_{j=1}^N f_j \exp(2\pi i(\overrightarrow{H}\cdot\overrightarrow{r_j}))),}
\end{equation}

где F$_j$ --- структурный фактор каждого из N симметрично независимых атомов в кристаллической ячейке, в котором закодирована информация об амплитуде (фактор атомного рассеяния) f$_j$ и фазе $\phi_j = 2\pi (hx_j+ky_j+lz_j)$ рассеяной этим атомом волны.

Стоит отметить, что данное выражение было получено, используя следующее предположение:

\begin{itemize}
\item Элементарная ячейка поделена на $N$ атомов в точках $\overrightarrow{r_j}$

\item Каждый из атомов рассеивает волну с амплитудой $f_j$  и фазой $\phi_j = 2\pi (hx_j+ky_j+lz_j)$.
\end{itemize}

Перейдем к более общему описанию --- заменим рассеяние от атомов на рассеяние от маленьких параллелепипедов, внутри которых находятся электроны. Тогда предыдущее приближение превращается в следующее:

\begin{itemize}
\item Элементарная ячейка поделена на $N$ маленьких параллелепипедов, объем каждого $\Delta V$ и позиция $\overrightarrow{r_j}$

\item Каждый из параллелепипедов рассеивает волну с амплитудой $\rho(\overrightarrow{r_j})\Delta V$  и фазой $\phi_j = 2\pi (hx_j+ky_j+lz_j) = 2\pi (\overrightarrow{H}\cdot\overrightarrow{r_j})$, где $\rho(\overrightarrow{r_j})$ --- электронная плотность внутри параллелепипеда.
\end{itemize}

Тогда выражение \ref{eq1} можно записать следующим образом, устремив объемы параллелепипедов к нулю:

\begin{equation}\label{eq2}
\mathrm{ 
F(h,k,l) = \sum\limits_{j=1}^N \rho(\overrightarrow{r_j})\Delta V_j \exp(2\pi i(\overrightarrow{H}\cdot\overrightarrow{r_j})) = \int\rho(\overrightarrow{r})\exp(2\pi i(\overrightarrow{H}\cdot\overrightarrow{r_j}))dV,}
\end{equation}

где интегрирование ведется по всему объему элементарной ячейки.

Можно заметить, что уравнение \ref{eq2} является обратным преобразованием Фурье, переводящее электронную плотность в структурные факторы. Тогда можно записать прямое преобразование Фурье:

\begin{equation}\label{eq21}
\mathrm{ 
\rho(\overrightarrow{r}) = \int F(h,k,l) \exp (-2\pi i (\overrightarrow{H}\cdot\overrightarrow{r}))dV^* = V^* \sum\limits_{(h,k,l)} F(h,k,l) \exp(-2\pi i (\overrightarrow{H}\cdot\overrightarrow{r_j}))},
\end{equation}

где V$^*$ --- объем элементарной обратной ячейки.

Таким образом, если амплитуда и фазы всех дифрагированных лучей были бы зарегистрированы, можно было бы рассчитать распределение электронной плотности в кристалле, применив преобразвование Фурье к дифракционной картине (рис. \ref{furie}). Однако в ходе рентгенодифракционного эксперимента регистрируются лишь интенсивность излучения, информация о фазах теряется, и для получения кристаллической структуры требуется решить так называемую "проблему фаз".

\begin{figure}[H]
	\centering
	\includegraphics[width=0.8\textwidth]{figures/furie.png}\hfill
	\caption{Схема связи электронной плотности и структурных факторов \cite{girolami_x-ray_2016}}
	\label{furie}
\end{figure}

\subsubsection{Прямые методы}

В работе \cite{hauptman_solution_1954} впервые было предложено решение проблемы фаз для центросимметричной группы P$\overline{1}$, которое требует только знание амплитуд структурных факторов и химического состава кристалла. Данный подход, основанный на вероятностном подходе и предположении, что в распределение амплитуд структурного фактора уже заложена информация о фазах, в дальнейшем был распространен на все центросимметричные группы, а затем адаптирован для нецентросимметричных структур. В дальнейшем этот метод и подобные были названы прямыми, их объединяют статистические взаимосвязи между двумя, тремя и четырьмя сильными отражениями --- структурные инварианты и полуинварианты \cite{hauptman_direct_1986}. 

Для того, чтобы избавиться от явной зависимости структурного фактора от угла рассеяния, авторы заменяют реальный кристалл с электронной плотностью $\rho(\overrightarrow{r})$ на идеальный, элементарная ячейка которого состоит из дискретных неподвижных точечных атомов, которые расположены в максимумах электронной плотности. Тогда структурный фактор $F(\overrightarrow{H})=|F(\overrightarrow{H})|\exp(i\phi(\overrightarrow{H}))$ следует заменить на нормализованный структурный фактор $E(\overrightarrow{H}) = |E(\overrightarrow{H})|\exp(i\phi(\overrightarrow{H}))$ \cite{giacovazzo_direct_1998}:

\begin{equation}\label{e}
	\mathrm{|E|^2 = \frac{|F|^2}{<|F|^2>}}
\end{equation}


В уравнении \ref{e} параметр $\textless|F|^2\textgreater$ есть математическое ожидание (среднее) квадрата амплитуды структурного фактора, для расчета которого нужна $a priori$ информация. Существует множество способов вычисления нормализованных структурных факторов, которые исходят из количества доступных данных, приведём некоторые из них \cite{giacovazzo_international_2010}:

\begin{enumerate}
	\item Отсутствие структурной информации: позиции атомов приняты случайными величинами. Пусть $\epsilon$~--- некоторый параметр, зависящий от группы симметрии, тогда:
	
	\begin{equation}\label{e1}
		\mathrm{<|F|^2> = \epsilon\sum\limits_{j=1}^N f_j^2}
	\end{equation}
	
	\item Известны $M$ групп по $M_i$ атомов в каждой, конфигурация которых известна, но неизвестны ориентация и позиция самих групп. Некоторое количество межатомных расстояний $r_{j_1j_2}$ известно, тогда для отражения $\overrightarrow{H}$:
	
	\begin{equation}\label{e2}
		\mathrm{<|F|^2> = \epsilon \left[\sum\limits_{j=1}^Nf_j^2 + \sum\limits_{i=1}^M\sum\limits_{j_1\neq j_2=1}^{M_i}f_{j_1}f_{j_2}\frac{\sin(2\pi |\overrightarrow{H}|r_{j_1j_2})}{2\pi |\overrightarrow{H}|r_{j_1j_2}}\right]}
	\end{equation}
	
	\item Известны $M$ групп по $M_i$ атомов в каждой, конфигурация и ориентация которых известны, но неизвестна позиция самих групп. Некоторое количество межатомных расстояний $r_{j_1j_2}$ зафиксированы, тогда для отражения $\overrightarrow{H}$:
	
	\begin{equation}\label{e3}
		\mathrm{<|F|^2> = \epsilon \left[\sum\limits_{j=1}^Nf_j^2 + \sum\limits_{i=1}^M\sum\limits_{j_1\neq j_2=1}^{M_i}f_{j_1}f_{j_2}\exp 2\pi i(\overrightarrow{H},\overrightarrow{r_{j_1j_2}})\right]}
	\end{equation}
	
	\item Известны $M$ групп атомов и их позиция, тогда:
	
	\begin{equation}\label{e4}
		\mathrm{<|F|^2> = |F_M|^2 + \epsilon\sum\limits_{i=1}^Qf_i^2},
	\end{equation}
	
	где $F_M$ --- структурный фактор известной подструктуры,, $Q$~--- количетсво неизвестных атомов.
\end{enumerate}

Однако чтобы рассчитать нормализованные структурные факторы по уравнению $\ref{e}$ требуется ещё два условия~--- величины должны быть в абсолютной шкале, а наблюдаемые амплитуды являются относительными величинами. Также в формулах для $<|F|^2>$ выше никак не учтено тепловое движение атомов. Оба обстоятельства можно преодолеть с использованием графика Вилсона \cite{wilson_determination_1942}, согласно которому наблюдаемые данные разделяются на несколько промежутков по $s = \frac{sin^2 \theta}{\lambda^2}$, в каждом из которых вычисляется средняя интенсивность $<I_{obs}> = <|F_{obs}|^2>$. Для каждого промежутка можно вычислить $K<I>$, где $K$~--- параметр, который необходим для перевода интенсивности рентгеновского излучения в абсолютные величины:

\begin{equation}\label{wils}
	\mathrm{K<I> = <|F_{obs}|^2>\exp(-2Bs^2)},
\end{equation}

где $B$~--- термический параметр, $<F_{obs}>$ вычисляется по уравнениям \ref{e1}, \ref{e2}, \ref{e3}, \ref{e4}.

Чтобы найти параметры $B$ и $K$, уравнение \ref{wils} логарифмируют, строят линейный график по уравнению \ref{wils2} в координатах $(s^2,\ln\frac{<I>}{<|F_{obs}|^2>}$ и получают параметры $B$ и $K$ после аппроксимации.

\begin{equation}\label{wils2}
	\mathrm{\ln\frac{<I>}{<|F_{obs}|^2>} = -\ln K-2Bs^2}
\end{equation}

Таким образом, после построения графика Вилсона и нахождения нужных параметров, нормализованные структурные факторы вычисляются по следующей итоговой формуле:

\begin{equation}
	\mathrm{|E|^2 = \frac{KI_{obs}}{<|F_{obs}|^2>\exp(-2Bs^2)}}
\end{equation}

Также известны \cite{giacovazzo_international_2010} плотности распределения полученных нормализованных структурных факторов, которые различаются в зависимости от симметрии кристаллической структуры (уравнения \ref{pe1}, \ref{pe2}), их вид представлен на рис. \ref{eimage}. Нетрудно показать, что математическое ожидание величины $(E^2-1)^2$, которое можно рассчитать как $<(E^2-1)^2> = \int\limits_0^\infty P(E)(E^2-1)^2dE$, для центросимметричных кристаллов больше (2.0), чем для нецентросимметричных (1.0), что можно использовать как критерий центросимметричности структуры.

\begin{equation}\label{pe1}
	\mathrm{\text{Центросимметричная: }P(|E|)=\sqrt{\frac{2}{\pi}}\exp(-\frac{E^2}{2})}
\end{equation}

\begin{equation}\label{pe2}
	\mathrm{\text{Нецентросимметричная: }P(|E|) = 2|E|\exp(-|E|^2)}
\end{equation}

\begin{figure}[H]
	\centering
	\includegraphics[width=0.5\textwidth]{figures/eimage.png}\hfill
	\caption{Плотность распределения $|E|$ для центросимметричных и нецентросимметричных кристаллов \cite{giacovazzo_international_2010}}
	\label{eimage}
\end{figure}




В работах \cite{sayre_squaring_1952}, \cite{cochran_relation_1952} была показана следующая взаимосвязь между высокоинтенсивными отражениями:

\begin{equation}\label{eq4}
\mathrm{ 
s(h+h', k+k', l+l') \approx s(h,k,l) s(h', k', l')}
\end{equation}


\begin{equation}\label{eq41}
\mathrm{
\phi(h+h', k+k', l+l')\approx \phi(h,k,l)+\phi(h',k',l')
}
\end{equation}

Из закона Фриделя (найти ссылку) следует, что:

\begin{equation}\label{eq5}
\mathrm{ 
\phi(-h-h',-k-k',-l-l') = -\phi(h+h', k+h', l+l')}
\end{equation}

Тогда, сложив уравнения \ref{eq5} и \ref{eq41} мы получим следующее выражение, которое получило название триплетное отношение:

\begin{equation}\label{eq6}
\mathrm{ 
\phi(h,k,l)+\phi(h',k',l') +\phi(-h-h',-k-k',-l-l')\approx 0}
\end{equation}

Из данного уравнения и закона Фриделя так же нетрудно получить выражение для квартета отражений:

\begin{equation}\label{eq7}
\mathrm{ 
\phi(h,k,l)+\phi(h',k',l') +\phi(h'',k'',l'')+\phi(-h-h'-h'',-k-k'-k'',-l-l'-l'')\approx 0}
\end{equation}


\subsubsection{Метод Паттерсона}

Функция Паттерсона $P(\overrightarrow{u})$ представляет собой автокорреляционную функцию (свёртку функции с собой) электронной плотности \cite{girolami_x-ray_2016}:

\begin{equation}
	\mathrm{P(\overrightarrow{u}) = \int_V \rho(\overrightarrow{r})\rho(\overrightarrow{r}+\overrightarrow{u})d\overrightarrow{r}},
\end{equation}

где $\overrightarrow{u}$ --- вектор координат ячейки Паттерсона, которая совпадает по размерам с обычной, $V$ --- объем кристаллической решетки в прямом пространстве.

Из свойств автокорреляционной функции прямо следует, что функция Паттерсона $P(\overrightarrow{u})$ принимает большие значения тогда и только тогда, когда электронная плотность принимает ненулевое значения в точке $\overrightarrow{r} = (x,y,z)$ и точке $\overrightarrow{r}+\overrightarrow{u} = (x+u, y+v, z+w)$, то есть в этих точках расположены атомы. Пики функции Паттерсона достигаются в таких точках ячейки Паттерсона, которые отвечает межатомным векторам кристалла.

Из экспериментальных данных функция Паттерсона рассчитывается следующим образом:

\begin{equation}\label{patt_start}
	\mathrm{P(\overrightarrow{u})=\frac{1}{V}}\sum_{\overrightarrow{H}}|F|^2(\overrightarrow{H})\exp(-2\pi i (\overrightarrow{H}, \overrightarrow{u}))
\end{equation}


Также можно выделить следующие свойства функции Паттерсона:

\begin{itemize}
	\item Функция Паттерсона всегда чётная, поскольку для каждой пары атомов существует пара межатомных векторов: $P(\overrightarrow{u}) = P(\overrightarrow{-u})$.
	\item Карта Паттерсона (графическое представление функции) обладает той же симметрией, что и группа Лауэ кристалла.
	\item Карта Паттерсона всегда имеет большой пик в начале координат --- явно следует из определения.
	\item Максимумы функции широкие и размазанные благодаря перекрыванию электронной плотности атомов.
\end{itemize}

Проанализируем количество максимумов функции Паттерсона \cite{rossmann_patterson_2001}. Пусть в элементарной ячейке $N$ атомов с атомными факторами рассеяния $f_j$. Из определения структурного фактора \ref{eq1} можно получить следующее выражение для $|F|^2$:

\begin{equation}
	\mathrm{|F(\overrightarrow{H})|^2 = F(\overrightarrow{H})F^*(\overrightarrow{H}) = \left[\sum\limits_{j=1}^N f_j \exp(2\pi i (\overrightarrow{H}, \overrightarrow{r_j}))\right]\left[\sum\limits_{j=1}^N f_j \exp(-2\pi i (\overrightarrow{H}, \overrightarrow{r_j}))\right]}
\end{equation}

\begin{equation}\label{patt1}
	\mathrm{|F(\overrightarrow{H})|^2 = \sum\limits_{j=1}^N f_j^2 + \sum\limits_{j=1}^N\sum\limits_{i=1, i\neq j}^N f_if_j\exp(2\pi i (\overrightarrow{H}, \overrightarrow{r_j}-\overrightarrow{r_i}))}
\end{equation}

Таким образом, объединив уравнения \ref{patt1} и \ref{patt_start}, получаем, что функция Паттерсона является суммой $N^2$ атомных взаимодействий, из которых $N$ в начале координат с весами $f_j^2$ Паттерсона и $N(N-1)$ попарных взаимодействий, пропорциональных $f_if_j$. Таким образом, в карте Паттерсона $N(N-1)$ максимумов, которые зависят от межатомных векторов и типов атомов в ячейке. Нетрудно показать, что интенсивность этих пиков пропорциональна атомным номерам соответствующих атомов $Z_i$ (уравнение \ref{patt_z}, $m$~--- фактор мультиплетности). Поскольку функция Паттерсона является четной, для описания уникальных попарных взаимодействий достаточно $N(N-1)/2$ значений. 

\begin{equation}
	\mathrm{P(\overrightarrow{u_{ij}}) = \frac{mZ_iZ_j}{\sum\limits_{j=1}^N f_j^2} }
\end{equation}

Поскольку $N$ максимумов карты Паттерсона, отвечающим свёртке электронной плотности каждого атома с самим собой, являются не очень информативными, от них можно избавиться, изменив структурные факторы перед расчётом функции Паттерсона по уравнению \ref{patt_mod_1}. Аналогично, если положения каких-то атомов заранее точно известны, пики, отвечающим их межатомным векторам, можно убрать благодаря модификации $|F(\overrightarrow{H})|^2$ в уравнении \ref{patt_mod_2}, где $r_1, r_2$ --- позиции атомов 1 и 2, $\sigma_1, \sigma_2$ --- их термические коэффициенты.

\begin{equation}\label{patt_mod_1}
	\mathrm{|F_{mod}(\overrightarrow{H})|^2 = |F(\overrightarrow{H})|^2 - \sum\limits_{j=1}^Nf_j^2}
\end{equation}

\begin{equation}\label{patt_mod_2}
	\mathrm{|F_{mod}(\overrightarrow{H})|^2 = |F(\overrightarrow{H})|^2 - \sum\limits_{j=1}^Nf_j^2\sigma_j^2-2f_1f_2\sigma_1\sigma_2\cos(2\pi (\overrightarrow{H},\overrightarrow{r_1}-\overrightarrow{r_2}))}
\end{equation}

Для борьбы с шириной пиков, которые могут сильно повлиять на корректность определение фаз дифракционных отражений, используют процедуру утончения карты Паттерсона. Для этого вместо структурных факторов и их интенсивностей $|F|^2$ используют нормализованные структурные факторы $|E|^2$. Поскольку интенсивность нормализованных структурных факторов не так сильно падает с увеличением угла рассеяния благодаря поправочному множителю, полученный набор получен как бы от атомов меньшего размера, в результате чего максимумы Паттерсона также будут более узкие.

Карту Паттерсона можно представить суммой нескольких копий исходной структуры с разными весами \cite{buerger_solution_1953}, которые различаются тем, какой атом находится в начале паттерсоновских координат $(u, v, w)$ (рис. \ref{patterson_image}). Для простых структур низкомолекулярных соединений возможно расшифровать карту, сопоставив каждому максимуму межатомный вектор. Однако количество пиков функции Паттерсона для структуры из $N$ атомов растёт по квадратичному закону, что делает невозможным прямую интерпретацию без дополнительных операций над картой.

\begin{figure}[H]
	\centering
	\includegraphics[width=0.8\textwidth]{figures/patterson.png}\hfill
	\caption{Схема карты Паттерсона для простейшей структуры из 3 атомов}
	\label{patterson_image}
\end{figure}

С развитием компьютерных технологий во второй половине 20 века произошел расцвет методов, основанных на функции Паттерсона. Так, особенно полезным для решения фазовой проблемы оказался метод суперпозиции карт Паттерсона \cite{hendrixson_locating_1997}. Было показано, что суперпозиция исходной карты Паттерсона со смещенной на какой-то трансляционный вектор может привести к меньшему набору межатомных векторов (рис. \ref{patt_superposition}), поскольку точки, соответствующие образу исходной структуры, будут всегда повторяться. Суперпозиция представляет собой, например, пересечение или сумму наборов векторов (см. далее). Через множество применений данной процедуры можно получить исходную структуру.

\begin{figure}[H]
	\centering
	\includegraphics[width=0.8\textwidth]{figures/patt_superposition.png}\hfill
	\caption{Суперпозиция (пересечение) исходной и смещенной на вектор $\overrightarrow{AB}$ карт Паттерсона}
	\label{patt_superposition}
\end{figure}

Есть три основных подхода, которые используются для выделения структуры молекулы из наложенных карт Паттерсона --- с помощью функций суммы, произведения и минимума суперпозиции \cite{rossmann_patterson_2001}. Пусть есть $N$ карт Паттерсона, каждая из которых смещена на вектор $\overrightarrow{u_i}$. Функция суммы (уравнение \ref{patt_sum}]) основывается на том, что она будет иметь наибольшие значения в точках пересечения всех карт, что будет отвечать расположениям атомов.

\begin{equation}\label{patt_sum}
	\mathrm{S(\overrightarrow{r}) = \sum\limits_{i=1}^N P(\overrightarrow{r}+\overrightarrow{u_i}) = \sum_{\overrightarrow{H}}}\left[|F(\overrightarrow{H})|^2 \exp (2\pi i (\overrightarrow{H}, \overrightarrow{r}))\left(\sum\limits_{j=1}^N\exp(2\pi i (\overrightarrow{H}, \overrightarrow{u_i}))\right)\right]
\end{equation}

Функция произведения суперпозиции (уравнение \ref{patt_prod}) является более сильной по сравнению с суммой в том смысле, что все точки, в которых нулевое значение хотя бы у одной из карт, будут обнулены.

\begin{equation}\label{patt_prod}
	\mathrm{Pr(\overrightarrow{r}) = \prod\limits_{i=1}^N P(\overrightarrow{r}+\overrightarrow{u_i})} 
\end{equation}
	
Если при наложении двух карт в точке ненулевая плотность, то суммарное значение функции Паттерсона будет больше, чем от одного изображения структуры. Если взять минимум от накладывающихся карт, то при удачной суперпозиции останется лишь плотность от одной копии структуры \cite{pavelcik_patterson-oriented_1992}. Так, в SHELXS-96 \cite{sheldrick_patterson_1997} имплементирован вариант суперпозиции карт Паттерсона с функцией минимума, которая берётся от двух копий утонченной функции Паттерсона, смещенных на векторы $\overrightarrow{u}$ и $\overrightarrow{-u}$.
	

\subsubsection{Метод обратного заряда (charge--flipping)}

В работе \cite{oszlanyi_ab_2004} был предложен простой, но эффективный алгоритм --- метод обратного заряда (charge--flipping), использующий прямое и обратное пространство. Данный алгоритм основан на итеративном приближении фаз дифракционных максимумов, которые задаются случаяными в начале, к настоящим фазам, позволящим определить кристаллическую структуру по дифракционным данным.

Пусть в эксперименте был зарегистрирован набор дифракционных отражений, которые характеризуются амплитудами $F_{obs}(\overrightarrow{H})$. Незарегистрированные отражения в других точках обратного пространства принимаются за нулевые. Алгоритм начинается с инициализации фаз зарегистрированных отражений, которые выбираются случайным образом так, чтобы выполнялся закон Фриделя: $\phi(-\overrightarrow{H}) = -\phi(\overrightarrow{H})$, тогда получаем набор структурных факторов $F = F_{obs}(\overrightarrow{H})\exp(i\phi(\overrightarrow{H}))$.

Запишем основные шаги алгоритма:

\begin{enumerate}
	\item Для набора структурных факторов $F = F_{obs}(\overrightarrow{H})$ рассчитаем распределение электронной плотности $\rho(\overrightarrow{r})$, применив обратное преобразование Фурье.
	\item Модифицируем электронную плотность $\rho(\overrightarrow{r})$ с помощью обращения заряда:
	
	\begin{equation}
		\mathrm{\rho(\overrightarrow{r})\geq \delta: g = \rho,}
	\end{equation}
	
	\begin{equation}
	\mathrm{\rho(\overrightarrow{r})<\delta: g = -\rho,}
	\end{equation}
		
	где $\delta > 0$ --- пороговое значение, параметр алгоритма.
	\item Переходим в обратное пространство, применив преобразование Фурье к модифицированной электронной плотности $g(\overrightarrow{r})$. Получаем набор структурных факторов $G = G(\overrightarrow{H})\exp(i\psi(\overrightarrow{H}))$.
	\item Ремодуляция: собираем новый набор структурных факторов $F$ следующим образом: $F = F_{obs}(\overrightarrow{H})\exp(i\psi(\overrightarrow{H}))$ --- амплитуды равняются экспериментальным, фазы берутся из набора $G$, полученного в ходе обращения заряда. Вернуться к шагу 1.
\end{enumerate}

\begin{figure}[H]
	\centering
	\includegraphics[width=0.8\textwidth]{figures/charge-flip.png}\hfill
	\caption{Схема итеративного цикла алгоритма charge--flipping}
	\label{chfl}
\end{figure}

В итерационный процесс (рис. \ref{chfl}) не заложено условие окончания алгоритма, для мониторинга процесса можно следить за метриками, такими как R-фактор. Главными достоинствами метода являются его простота и отсутствие необходимости в дополнительных данных --- метод позволил получить структуры множества соединений $ab initio$ при наличии лишь экспериментальных дифракционных данных высокого разрешения, задавая лишь пороговое значение $\delta$ для электронной плотности.

В следующей работе \cite{oszlanyi_it_2005} авторы модифицировали четвертый шаг алгоритма, явно используя отражения низкой интенсивности. Перед запуском алгоритма теперь дифракционные максимумы сортируются по наблюдаемой амплитуде и размечаются на две группы, к которым будут применяться разные преобразования в ходе вычислений --- сильные и слабые отражения. Для сильных всё остается без изменений --- фаза остаётся из шага 2, амплитуда структурного фактора берётся из экспериментальных данных. Фаза слабых отражений дополнительно смещается на $\Delta\phi$, а их амплитуда не изменяется на наблюдаемую. Это значит, что экспериментальные данные слабых отражений не используются в алгоритме, кроме начальной разметки на группы.

В модифицированном алгоритме charge--flipping появилось два дополнительных параметра: сдвиг фазы $\Delta\phi$ и доля отражений, которые можно считать слабыми, принятая за 20\%. Авторы показали, что оптимальное значение $\Delta\phi = \frac{\pi}{2}$ --- поскольку при сдвиге на такую величину волны слабых отражений заменяются на ортогональные изначальным. Такое возмущение волн распространения слабых максимумов позволило на порядок увеличить долю успешных решений структур по сравнению с начальным алгоритмом.

Описанная модификация происходит в обратном пространстве, и в работе \cite{oszlanyi_charge_2008} авторы описывают подход к улучшению алгоритма с помощью изменения процедуры обработки функции электронной плотности. Charge--flipping в его классическом варианте можно рассматривать как алгоритм локального возмущения низких плотностей, то есть он никак не влияет на области с высокими значениями электронной плотности. Если в начале выполнения расчета будут получены некорректные высокие значения электронной плотности, то их практически невозможно исправить тривиальным обращением знака. В качестве аналога использования слабых отражений предлагается следующее улучшение в прямом пространстве: пусть в $(n+1)$--й цикл алгоритма:


\begin{equation}
	\mathrm{\rho^n(\overrightarrow{r}) < \delta: g^{n+1}(\overrightarrow{r}) = -\rho^n(\overrightarrow{r})},
\end{equation}
 
 
\begin{equation}
	\mathrm{\rho^n(\overrightarrow{r}) \geq \delta: g^{n+1} = \rho^n + \beta(\rho^n-\rho^{n-1})}
\end{equation} 
  
где обычно $\beta\in[0.5,1.0]$. Суть операции в следующем: измененная электронная плотность $g^{n+1}$ будет вне интервала, сформированными $\rho^{n-1}, \rho^{n}$, со стороны последней. Процедура получила название 'flip--mem' и значительно улучшает классический алгоритм.

В этой же работе отмечено, что использование нормализованных структурных факторов $E(\overrightarrow{H}) = \frac{F_{obs}(\overrightarrow{H})}{[\sum_j f_j^2(\overrightarrow{H})]^{1/2}}$ вместо стандартных амплитуд позволяет увеличить скорость сходимости для больших структур. Кроме того, все модификации, которые были ранее упомянуты, подходят и для варианта с нормализованными амплитудами. Использование $E$ уменьшает на порядок число итераций до сходимости по сравнению с $F$ для всех вариантов алгоритма.

Метод обратного заряда позволяет в некоторых случаях решать макромолекулярные структуры $ab initio$ \cite{dumas_macromolecular_2008}. Так, для таких структур, как лизоцим (2385 атомов, $d_{min} = 1.1\angstrom$), алкогольдегидрогеназа(5866 атомов, $d_{min} = 1.0\angstrom$), апамин (385 атомов, $d_{min} = 0.95\angstrom$), фазы, рассчитанные алгоритмом, позволили корректно определить структуру. Авторы использовали метод в варианте с нормализованными структурными факторами $E(\overrightarrow{r})$, использованием слабых отражений (доля слабых отражений $\omega = 0.1$). Также авторы продемонстрировали, что charge--flipping является эффективным инструментов для нахождения фаз для рентгенодифракционных данных низкого разрешения сложных структур с тяжелыми атомами, а такжюе с значимым анормальным рассеянием.

Для следующей модификации обратного пространства требуется ввести понятие формула тангенсов. Квазинормализованные структурные факторы $\epsilon(\overrightarrow{H})$ \cite{karle_unified_1959} можно рассчитать как:

\begin{equation}\label{eqquazi}
	\mathrm{ \epsilon_(\overrightarrow{H}) = \frac{1}{\sigma_2^{1/2}}\sum\limits_{j=1}^Nf_{j}(\overrightarrow{H})\exp(2\pi i (\overrightarrow{H}, \overrightarrow{r})) = |\epsilon(\overrightarrow{H})|\exp(i\phi(\overrightarrow{H})),
		}
\end{equation}

\begin{equation}\label{sigma}
	\mathrm{ \sigma_2 = \sum\limits_{j=1}^N f_j^2(\overrightarrow{H})
	}
\end{equation}

Квазинормализованные структурные факторы, хоть и для большого числа отражений совпадают, следуют различать от нормализованных. С помощью них была получена \cite{karle_symbolic_1966} следующее равенство для определения фаз прямыми методами, называемая формула тангенсов:

\begin{equation}\label{sigma2}
	\mathrm{\tan\phi(\overrightarrow{H}) = \frac{\sum\limits_{\overrightarrow{K}}|E(\overrightarrow{K})E(\overrightarrow{H}-\overrightarrow{K})|\sin(\phi(\overrightarrow{K})+\phi(\overrightarrow{H}-\overrightarrow{K}))}{\sum\limits_{\overrightarrow{K}}|E(\overrightarrow{K})E(\overrightarrow{H}-\overrightarrow{K})|\cos(\phi(\overrightarrow{K})+\phi(\overrightarrow{H}-\overrightarrow{K}))}
	}
\end{equation}

В работе \cite{coelho_charge-flipping_2007} авторы представили свой вариант алгоритма обратного заряда с использованием нормализованных структурных факторов и новым способом возмущения, который применяет знание о формуле тангенсов, связывающей три отражения:

\begin{equation}\label{sigma3}
	\mathrm{\tan\phi_{tf}(\overrightarrow{H}) = \frac{\sum\limits_{\overrightarrow{K}}|E(\overrightarrow{K})E(\overrightarrow{H})E(\overrightarrow{H}-\overrightarrow{K})|\sin(\phi(\overrightarrow{K})+\phi(\overrightarrow{H}-\overrightarrow{K}))}{\sum\limits_{\overrightarrow{K}}|E(\overrightarrow{K})E(\overrightarrow{H})E(\overrightarrow{H}-\overrightarrow{K})|\cos(\phi(\overrightarrow{K})+\phi(\overrightarrow{H}-\overrightarrow{K}))}
	}
\end{equation}

Итоговый алгоритм (после случайной расстановки фаз для зарегистрированных отражений) выглядит следующим образом:

\begin{enumerate}
	\item Обнулить 50\% структурных факторов $E(\overrightarrow{H})$ с самыми низкими значениями амплитуд. Модуль остальных структурных факторов равен наблюдаемым.
	\item Рассчитать электронную плотность $\rho(\overrightarrow{r})$ с помощью обратного преобразования Фурье.
	\item Отнормировать электронную плотность, чтобы максимум функции был равен единице.
	\item Определить граничное значение $\delta$ так, что 60\% значений $\rho(\overrightarrow{r})$ лежат ниже порога.
	\item Преобразовать плотность:
	
	\begin{equation}
		\mathrm{\rho<\delta: g = -\rho}
	\end{equation}
	 
	\begin{equation}
		\mathrm{\rho\geq\delta: g = \delta + (\rho-\delta)^{1/2}}
	\end{equation}

	\item Рассчитать набор структурных факторов $G$, применив преобразование Фурье к электронной плотности $g(\overrightarrow{r})$
	\item Добавить к фазам с наибольшими значениями $E(\overrightarrow{H})$ долю разницы между фазами, полученной обратным зарядом $\phi_{cf}$ и рассчитанной по формуле тангенсов $\phi_{tf}$:
	
	\begin{equation}
		\mathrm{\phi(\overrightarrow{H}) = \phi_{cf}(\overrightarrow{H})+\alpha(\overrightarrow{H})(\phi_{tf}(\overrightarrow{H})-\phi_{cf}(\overrightarrow{H})),}
	\end{equation}
	 
	 где $\alpha(\overrightarrow{H})$ --- параметр, рассчитываемый алгоритмом для каждого отражения, который отражает достоверность рассчитанных в ходе выполнения программы фаз. Вернуться к шагу 1.
\end{enumerate}

Авторы продемонстрировали, что предложенный вариант метода позволяет решать за несколько минут структуры, которые ранее требовали сотни тысяч итераций. Использование формулы тангенсов для высокоинтенсивных отражений позволило повысить устойчивость и эффективность алгоритма, особенно при работе с низким разрешением данных (более 1\angstrom), где классический вариант алгоритма чаще всего терпит неудачу.

В работе \cite{van_der_lee_charge_2013} был проведен сравнительный анализ метода обратного заряда с другими стандартными инструментами рутинного определения кристаллографических структур низкомолекулярных соединений. Для этого автор использовал charge--flipping в реализации программы SUPERFLIP \cite{palatinus_it_2007} и традиционные прямые методы --- SHELXS, SHELX86, SHELXD, SIR2004 \cite{sheldrick_shelxt_2015}. Тестирование проводилось на наборе данных из 518 структур, включащих в себя органические, металлорганические и неорганические соединения. Метод обратного заряда показал эффективность, сравнимую с прямыми методами, в среднем процент успешного решения структур составляет более 92\%. На рис. \ref{charge_direct} представлены значения среднего R--фактора, достигаемого в ходе уточнения после успешного решения структур каждым методом. Также в статье показано, что charge--flipping является самым быстрым методом из рассматриваемых при той же эффективности решения. Таким образом, рассматриваемый алгоритм является полностью пригодным для определения низкомолекулярных структур в качестве рутинного метода.


\begin{figure}[H]
	\centering
	\includegraphics[width=0.5\textwidth]{figures/charge_direct.png}\hfill
	\caption{Средний R--фактор успешных решений различных методов решения}
	\label{charge_direct}
\end{figure}

\subsubsection{Метод VLD}

Следующий рассматриваемый метод под названием Vive La Diff\'erence (VLD) является эффективным и универсальным методом для определения кристаллических структур, основанным на преобразовании электронной плотности и разностном Фурье--синтезе исходя из вероятностных и статических зависимостей между фазами и амплитудами \cite{burla_random_2010}. 

Как и charge--flipping, метод является итеративным, но более комплексным, чем метод обратного заряда. Перед стартом расчёта фаз всем наблюдаемым амплитудам нормализованных структурных факторов $E_{obs}$ присваиваются случайные фазы $\phi_m$ и рассчитывается электронная плотность кристаллической ячейки $\rho_m$. Здесь и далее величины, определяемые в ходе выполнения расчёта, назовём модельными. Алгоритм состоит из следующих шагов (схема представлена на рис. \ref{vld}):

\begin{enumerate}
	\item Модифицируем электронную плотность $\rho(\overrightarrow{r})$ следующим образом: обнуляем функцию во всех точках, кроме 2.5\% с наибольшим значением плотности. По полученной измененной электронной плотности $g(\overrightarrow{r})$ рассчитываем модельные нормализованные структурные факторы $E_m$ с помощью Фурье--преобразования.
	\item Вычисляем разностную электронную плотность --- синтез Фурье со следующими коэффициентами для каждого отражения:
	
	\begin{equation}
		\mathrm{\Delta E = (mE_{obs} - E_m)\exp(i\phi_m)}
	\end{equation}
		
	где $m$ --- коэффициент корреляции между модельными и реальными фазами.
	
	\item Модифицируем полученную разностную плотность: обнуляем функцию во всех точках, кроме 4\% с наибольшим значением по модулю. С помощью Фурье--преобразования рассчитываем разностные структурные факторы с амплитудами и фазами $E_{diff}, \phi_{diff}$.
	
	\item Используем полученные из разностной электронной плотности структурные факторы для расчёта коэффициента Фурье--синтеза (шаг 2), рассчитываем новую разностную плотность. Повторяем настоящий и предыдущий шаги $\beta$ раз, итоговые параметры структурных факторов $E_{diff}', \phi_{diff}'$.
	
	\item С помощью формулы тангенсов, которая была получена из вероятностного анализа и распределения фон Мизеса, можно рассчитать новые фазы $\phi_{calc}$:
	
	\begin{equation}
		\mathrm{\tan(\theta) = \frac{E_m\sin\phi_m+\omega_{diff}E_{diff}\sin\phi_{diff}}{E_m\cos\phi_m+\omega_{diff}E_{diff}\cos\phi_{diff}}}
	\end{equation}
	
	где $\omega_m$ --- параметр сходства модельной и реальной структур, отражающий среднюю корреляцию между моделью и структурой.
	
	\item Экспериментальные $E_{obs}$ и рассчитанные на предыдущем шаге фазы $\phi_{calc}$ используются для расчета электронной плотности, которая претерпевает $\gamma$ циклов модификации, как в шаге 1. 
	
	\item Рассчитывается метрика $RESID$ --- средняя ошибка между наблюдаемыми и рассчитанными амплитудами:
	
	\begin{equation}
		\mathrm{RESID = \frac{\sum_{\overrightarrow{H}}|E_{obs} - E_m|}{\sum_{\overrightarrow{H}}E_{obs}}}
	\end{equation}

	Если значение меньше 0.3 --- можно считать, что решение достигнуто и цикл останавливается. Иначе --- возвращение к шагу 1.
\end{enumerate}


\begin{figure}[H]
	\centering
	\includegraphics[width=0.7\textwidth]{figures/vld.png}\hfill
	\caption{Схема итеративного цикла алгоритма VLD}
	\label{vld}
\end{figure}

Подробности процедур выбора настраиваемых параметров алгоритма и рассчета критических параметров, как $\omega_m$ и $m$, подробно описаны в оригинальной работе. Авторы протестировали предложенный метод на 33 низкомолекулярных структурах, разных по пространственной группе симметрии и наличию тяжелых атомов, из которых VLD успешно решил 30 структур за не более, чем 300 циклов. При дополнительных запусках с другими случайными фазами при инициализации модели привели к решению всего набора данных. Таким образом, показано, что метод является достаточно стабильным и обладает быстрой сходимостью (менее, чем 1 минута). Важно отметить, что в отличие от метода обратного заряда, метод решает структуру в правильной группе симметрии, а не P1.

В следующей работе \cite{burla_advances_2011} авторы развивают идеи первой работы и описывают существенные усовершенствования оригинального подхода к решению кристаллографической фазовой задачи. Ключевым усовершенствованием является внедрение процедуры RELAX, который позволяет автоматически перместить правильно ориентированную, но смещенную модель в корректное базисную позицию в пространственной группе. RELAX основана на наблюдении, что часто прямые методы определяет молекулярные фрагменты которые корректно ориентированы, но неправильно расположены. Использование процедуры существенно повысило процент успешно решённых структур, особенно в случае макромолекулярных соединений. Также в статье представлено значительное количество улучшений, связанные с улучшенной оценкой параметров качества модели, оптимизацией этапов модификаций электронной плотности (добавлена адаптивность модификации), а также дополнительные параметры качества решений для управления остановкой VLD. Тестирование на малых, средних молекулах и белках (разрешение до 1.2 $\angstrom$) показало, что доработанный алгоритм решает структуры быстрее в среднем в 3--6 раз, определяет структуры белков за время, сравнимое с прямыми методами (SIR2011), а добавление RELAX позволило находить решения для крупных молекул именно благодаря этой процедуре.

В публикации \cite{burla_phasing_2011} представлено множество дополнительных вариантов алгоритма VLD (4 протокола), которые ориентированы на решение проблемы фаз для среднемолекулярных и белковых соединений. Вариации метода различаются подходами к контролю метрик сходимости и внутренних параметров, а также способами комбинирования модели, разностной и обычной электронных плотностей (количество преобразований Фурье, использование или игнорирование экспериментальных данных, а также отказ от формулы тангенсов). В ходе оценки эффективности показано, что новые подходы уступают по успешности и скорости решения варианту с процедурой RELAX, но могут быть в дальнейшем оптимизированы для $ab initio$ решения макромолекулярных соединений.





Проблема фаз – определение фаз структурных факторов - давно является одной из главных задач рентгеновской кристаллографии. На сегодняшний день существуют решения, такие как прямые [1,2], VLD (vive la difference) [3] и метод обратного заряда (charge-flipping) [4], но они, как правило, ограничены атомным разрешением рентгенодифракционных данных. Эти методы требуют полного набора интенсивностей дифракционных максимумов и часто требуют выращивания высококачественных образцов кристаллов, что может быть сложной задачей, особенно для слабо рассеивающих кристаллов или больших молекул, таких как белки [5]. Для определения структур макромолекулярных соединений разработаны методы молекулярного замещения и изоморфного замещения, требующие дополнительную информацию - знание о структуре белка с той же аминокислотной последовательностью или результаты рентгенодифракционных экспериментов той же структуры с добавлением тяжелых атомов, соответственно [6]. Таким образом, решение проблемы фаз является особенно актуальной задачей белковой кристаллографии из-за отсутствия ab initio решений. 

Применение машинного обучения в области рентгеновской дифракции началось совсем недавно и сейчас бурно развивается. Чаще всего авторы пытаются обойти решение проблемы фаз, предсказывая кристаллическую структуру из доступных экспериментальных данных. Так, были разработаны модели глубокого обучения, позволяющие миновать проблему фаз, работая напрямую с функцией Паттерсона, которые получаются из дифракционных данных, не требуя информации о фазах отражений [7]. Эти модели представляют собой сверточные нейронные сети для предсказания распределения электронной плотности структуры по функции Паттерсона, демонстрируя многообещающие результаты на простых примерах дипептидов и обладают несомненным потенциалом для более широкого применения в белковой кристаллографии.
В недавней статье [8] с помощью методов глубокого обучения авторы предприняли попытку решить проблему фаз. В качестве объектов предсказания они выбрали центросимметричные структуры, для которых фазы отражений принимают только два возможных значения. Авторы представили нейронную сеть, представляющую собой бинарный классификатор из блоков трехмерных свёрток и многослойных перцептронов. Обучение проводилось на синтетических кристаллических молекулярных структурах. Этот подход продемонстрировал способность решать фазовую задачу с разрешением всего 2Å, используя всего 10-20% данных, требуемых традиционными методами. Также в работе реализована идея рециклинга фаз (phase recycling) - исходные данные прогоняются несколько раз через модель, что увеличивает точность классификации. Таким образом, впервые был продемонстрирован потенциал машинного обучения для решения проблемы фаз, но только для центросимметричных структур.

На основе методов глубокого обучения были также предложены решения аналогичной задачи в физике, особенно в области когерентной безлинзовой микроскопии. В обзорной статье [9] выделяются три подхода: DL-post-processing, где уточняются «плохие фазы», полученные из исходных интенсивностей; DL-in-processing, в котором нейронная сеть используется для непосредственного расчета фаз из интенсивностей; и DL-pre-processing, при котором обученная модель повышает разрешение микроскопического изображения, а фазы из полученного изображения определяются детерминированными методами. Однако, в кристаллографии дифракционные данные имеют гораздо меньшее разрешение, чем в когерентной микроскопии, что исключает возможность прямого применения существующих решений.
Таким образом, решение проблемы фаз белковой кристаллографии является актуальной задачей, нерешаемой рутинными методами. Инструменты на основе методов глубокого обучения способны преодолеть ограничения традиционных подходов, позволяя определять кристаллические структуры на основе ограниченных данных с низким разрешением. По мере развития исследований в этой области, вероятно, алгоритмы глубокого обучения станут незаменимым инструментом в области кристаллографии, облегчая решение сложных структур, которые ранее были неразрешимыми. Таким образом, разработка методов решения проблемы фаз с помощью ИИ является целью работы.