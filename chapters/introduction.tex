\section*{Введение}
\addcontentsline{toc}{section}{Введение}

Проблема фаз~--– определение фаз структурных факторов~--- давно является одной из главных задач рентгеновской кристаллографии. На сегодняшний день существуют такие решения, как прямые методы, методы Паттерсона и обращенного заряда (charge--flipping), однако они, как правило, ограничены атомным разрешением рентгенодифракционных данных. Данные методы подразумевают использование полного набора интенсивностей дифракционных максимумов и часто требуют выращивания высококачественных образцов кристаллов, что является сложной задачей для слабо рассеивающих кристаллов или больших молекул, таких как белки. Для определения структур макромолекулярных соединений разработаны методы молекулярного замещения и изоморфного замещения, требующие дополнительную информацию - знание о структуре кристаллов других белков с той же аминокислотной последовательностью или результаты рентгенодифракционных экспериментов с добавлением тяжелых атомов в исследуемую структуру. 

Применение машинного обучения в области рентгеновской дифракции сейчас бурно развивается. Чаще всего авторы пытаются обойти решение проблемы фаз, предсказывая кристаллическую структуру из доступных экспериментальных данных. Однако на сегодняшний день уже показано, что с помощью методов искусственного интеллекта можно предсказывать фазы дифракционных отражений центросимметричных кристаллов, что позволяет решить фазовую проблему с атомным разрешением всего 2 \text{\AA}, используя всего 10--20\% данных, требуемых традиционными методами. Таким образом, методы машинного обучения способны преодолеть проблему фаз, снимая ограничение на высокое разрешение рентгенодифракционных данных.

Целью нашей работы стала разработка подходящих средств решения проблемы фаз с помощью методов искусственного интеллекта и анализ их эффективности решения проблемы фаз для нецентросимметричных структур.




%Решение проблемы фаз является важной задачей рентгеноструктурного анализа, особенно актуальной для белковой кристаллографии ввиду отсутствия ab initio решений в этой области. Методы машинного обучения способны преодолеть данную задачу, снимая ограничение на высокое разрешение рентгенодифракционных данных.

%В работе разработан генератор синтетических рентгенодифракционных данных, который может быть использован для решения прикладных задач методами ИИ, и автоматизированный конвейер для проведения воспроизводимых экспериментов для решения задачи в рамках предложенной методологии. Был предложен подход, заключающийся в предсказывании моделью машинного обучения дополнительных дифракционных максимумов, используя лишь структурные факторы известных отражений, тем самым увеличивая разрешение.  Также были разработаны модели FFT\_UNet и XRD\_Transformer, учитывающие специфику задачи; проведены сравнительный анализ их работы с помощью GradCAM и связей внимания и интерпретация на реальных данных. 
