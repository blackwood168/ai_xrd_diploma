\section*{Введение}
\addcontentsline{toc}{section}{Введение}

Решение проблемы фаз является важной задачей рентгеноструктурного анализа, особенно актуальной для белковой кристаллографии ввиду отсутствия ab initio решений в этой области. Методы машинного обучения способны преодолеть данную задачу. Предлагается увеличивать разрешение дифракционной картины, предсказывая моделью машинного обучения дальние отражения по ближним, что позволит решить проблему фаз ab initio для биомолекул. В работе разработан генератор синтетических рентгенодифракционных данных, который может быть использован для решения прикладных задач методами ИИ, и пайплайн для проведения воспроизводимых экспериментов для решения задачи в рамках предложенной методологии. Также были разработаны модели FFT\_UNet и XRD\_Transformer, подходящие под специфику задачи; проведены их сравнительный анализ и интерпретация их работы с помощью GradCAM и связей внимания. Было показано и обосновано, что методы глубокого обучения способны считывать кристаллографические связи и законы в рентгенодифракционных данных, но их точности численного восстановления амплитуд структурных факторов не хватает для решения проблемы фаз в рамках предложенной методологии. 
