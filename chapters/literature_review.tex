\section{Обзор литературы}

\subsection{Рентгеновская кристаллография}


Рентгеновская кристаллография является важнейшим инструментом определения трехмерных структур кристаллов. Этот метод позволяет получить бесценные сведения об атомных и молекулярных структурах кристаллов, что крайне важно для понимания свойств и функций материалов в различных областях, включая химию, биологию и материаловедение. Рентгеноструктурный анализ основан на упругом рассеянии монохроматического рентгеновского излучения на трехмерной регулярной решетке атомов твердого вещества, что приводит к интерференции рентгеновских лучей.

Дифракция в кристалле описывается как отражения от семейств параллельных кристаллографических плоскостей элементарной ячейки кристалла. Каждая точка дифракционной картины задается набором из индексов Миллера $h,k,l$. Любое отражение имеет математическое представление, известное как структурный фактор $F(h,k,l)$, зависящий от расположения и коэффициентов рассеяния всех
атомов в элементарной ячейке кристалла. Структура кристалла получается при расчете электронной плотности с помощью Фурье-преобразования: \[\rho(x, y, z) = V^*\sum_{h,k,l} F(h,k,l) exp[(-2\pi i (hx+ky+lz)],\] где $V^*$~-- объем элементарной обратной ячейки \cite{giro}. 

\subsection{Проблема фаз}

Дифракция в кристалле описывается как отражения от семейств параллельных кристаллографических плоскостей элементарной ячейки кристалла. Каждая точка дифракционной картины задается набором из индексов Миллера $h,k,l$. Любое отражение имеет математическое представление, описывающее общую рассеянную от всех атомов элементарной ячейки волну, известное как структурный фактор $F(h,k,l)$:

\begin{equation}\label{eq1}
\mathrm{ 
F(h,k,l) = \sum\limits_{j=1}^N F_j (h,k,l) = \sum\limits_{j=1}^N f_j \exp(2\pi i(hx_j+ky_j+lz_j)),}
\end{equation}

где F$_j$ --- структурный фактор каждого из N симметрично независимых атомов в кристаллической ячейке, в котором закодирована информация об амплитуде (фактор атомного рассеяния) f$_j$ и фазе $\phi_j = 2\pi (hx_j+ky_j+lz_j)$ рассеяной этим атомом волны.

Из формулы \ref{eq1} следует, что структурный фактор является комплексной величиной. Проблемой фаз называют задачей определения фазы структурных факторов $F(h,k,l)$, зная их амплитуды $|F(h,k,l)|$ для значительного количества отражений, доступных из эксперимента. Стоит отметить, что данное выражение было получено со следующими предположениями:

\begin{itemize}
\item Элементарная ячейка поделена на $N$ атомов $\overrightarrow{r_j} = x_j\overrightarrow{a} + y_j\overrightarrow{b} + z_j\overrightarrow{c}$

\item Каждый из параллелепипедов рассеивает волну с амплитудой $f_j$  и фазой $\phi_j = 2\pi (hx_j+ky_j+lz_j)$.
\end{itemize}

Подойдем к описанию дифракции рентгеновской волны более общим подходом --- заменим рассеяние от атомов на рассеяние от маленьких параллелепипедов, внутри которых находятся электроны. Тогда предыдущие предположения трансформируются в следующие:

\begin{itemize}
\item Элементарная ячейка поделена на $N$ маленьких параллелепипедов, каждый объема $\Delta V$ и в позиции $\overrightarrow{r_j} = x_j\overrightarrow{a} + y_j\overrightarrow{b} + z_j\overrightarrow{c}$

\item Каждый из параллелепипедов рассеивает волну с амплитудой $\rho(\overrightarrow{r_j})\Delta V$  и фазой $\phi_j = 2\pi (hx_j+ky_j+lz_j)$, где $\rho(\overrightarrow{r_j})$ --- электронная плотность внутри параллелепипеда.
\end{itemize}

Тогда выражение \ref{eq1} можно записать следующим образом, устремив объемы параллелепипедов к нулю:

\begin{equation}\label{eq2}
\mathrm{ 
F(h,k,l) = \sum\limits_{j=1}^N \rho(\overrightarrow{r_j})\Delta V_j \exp(2\pi i(hx_j+ky_j+lz_j)) = \int\rho(\overrightarrow{r})\exp(2\pi i(hx_j+ky_j+lz_j))dV,}
\end{equation}

где интегрирование ведется по всему объему элементарной ячейки.

Можно заметить, что уравнение \ref{eq2} является обратным преобразованием Фурье, переводящее электронную плотность в структурные факторы. Тогда можно записать прямое преобразование Фурье:

\begin{equation}\label{eq21}
\mathrm{ 
\rho(\overrightarrow{r}) = \int F(h,k,l) \exp (-2\pi i (hx+ky+lz))dV^* = V^* \sum\limits_{(h,k,l)} F(h,k,l) \exp(-2\pi i \overrightarrow{H}\overrightarrow{r})},
\end{equation}

где V$^*$ --- объем элементарной обратной ячейки.

Таким образом, без решения проблемы фаз невозможно рассчитать функцию электронной плотности электронную плотность и, как следствие, получить структуру кристалла.

\subsection{Прямые методы}

В работе \cite{hauptman_solution_1954} впервые было предложено решение проблемы фаз для центросимметричной группы P$\overline{1}$, которое требует только знание амплитуд структурных факторов и химического состава кристалла. Данная рутинный подход в дальнейшем был распространен на все центросимметричные группы, а затем адаптирован для нецентросимметричных структур. В дальнейшем этот метод и подобные были названы прямыми, их объединяют статистические взаимосвязи между двумя, тремя и четырьмя сильными отражениями \cite{hauptman_direct_1986}. 

Поскольку для центросимметричных кристаллов значения фаз структурных факторов принимает 0 или $\pi$ \cite{cowtan_phase_2003}, можно ввести функцию знака структурного фактора $s(h,k,l) = \pm 1$. Заметим, что значения атомных координат $x_j, y_j, z_j$ зависят от выбора центра ячейки. Для группы P$\overline{1}$ существует 8 возможных положений, которые соответствуют позициям центров инверсии. Значит, значение фазы зависит не только от структуры, но также от выбора центра.

Пусть центр ячейки смещен на вектор $\overrightarrow{\epsilon} = (\epsilon_1, \epsilon_2, \epsilon_3),$ где $\epsilon_j \in {0, \frac{1}{2}}, j \in {1,2,3}$. Тогда:

\begin{equation}\label{eq3}
\mathrm{ 
F'(h,k,l) = \sum\limits_{j=1}^N f_j \exp(2\pi i\overrightarrow{H}(\overrightarrow{r_j}+\overrightarrow{\epsilon})) = F(h,k,l) \exp(2\pi i\overrightarrow{H}\overrightarrow{\epsilon}) = F(h,k,l) \cos (2\pi \overrightarrow{H}\overrightarrow{\epsilon})}
\end{equation}

Заметим, что знак структурного фактора $F(h,k,l)$ зависит от структуры и не зависит от выбора центра ячейки, если каждый из индексов Миллера четный. Знак любого структурного фактора $F(h_1,k_1,l_1)$, где $h_1$ --- нечетный, $k_1$ и $l_1$ --- четные, можно определить самостоятельно, однако, тогда знак всех других структурных факторов $F(h,k,l)$, где $h_1$ --- нечетный, $k_1$ и $l_1$ --- четные, может измениться при смене базиса и зависит только от кристаллической структуры. И так далее можно рассмотреть остальные комбинации четных и нечетных индексов. Введем 8 классов четности дифракционных отражений: (ччч), (ччн), (чнч), (нчч), (чнн), (нчн), (ннч), (ннн), которые отражают различные комбинации четных и нечетных индексов отражений. Тогда, как уже было сказано ранее, для отражений класса (ччч) структурный фактор не меняется из-за изменения базиса, для остальных --- может измениться, поэтому мы можем сами выбирать знак.

Также можно ввести нормализованные структурные факторы \cite{giacovazzo_direct_1998}


В работах \cite{sayre_squaring_1952}, \cite{cochran_relation_1952} была показана следующая взаимосвязь между высокоинтенсивными отражениями:

\begin{equation}\label{eq4}
\mathrm{ 
s(h+h', k+k', l+l') \approx s(h,k,l) s(h', k', l')}
\end{equation}


\begin{equation}\label{eq41}
\mathrm{
\phi(h+h', k+k', l+l')\approx \phi(h,k,l)+\phi(h',k',l')
}
\end{equation}

Из закона Фриделя (найти ссылку) следует, что:

\begin{equation}\label{eq5}
\mathrm{ 
\phi(-h-h',-k-k',-l-l') = -\phi(h+h', k+h', l+l')}
\end{equation}

Тогда, сложив уравнения \ref{eq5} и \ref{eq41} мы получим следующее выражение, которое получило название триплетное отношение:

\begin{equation}\label{eq6}
\mathrm{ 
\phi(h,k,l)+\phi(h',k',l') +\phi(-h-h',-k-k',-l-l')\approx 0}
\end{equation}

Из данного уравнения и закона Фриделя так же нетрудно получить выражение для квартета отражений:

\begin{equation}\label{eq7}
\mathrm{ 
\phi(h,k,l)+\phi(h',k',l') +\phi(h'',k'',l'')+\phi(-h-h'-h'',-k-k'-k'',-l-l'-l'')\approx 0}
\end{equation}

Проблема фаз – определение фаз структурных факторов - давно является одной из главных задач рентгеновской кристаллографии. На сегодняшний день существуют решения, такие как прямые [1,2], VLD (vive la difference) [3] и метод обратного заряда (charge-flipping) [4], но они, как правило, ограничены атомным разрешением рентгенодифракционных данных. Эти методы требуют полного набора интенсивностей дифракционных максимумов и часто требуют выращивания высококачественных образцов кристаллов, что может быть сложной задачей, особенно для слабо рассеивающих кристаллов или больших молекул, таких как белки [5]. Для определения структур макромолекулярных соединений разработаны методы молекулярного замещения и изоморфного замещения, требующие дополнительную информацию - знание о структуре белка с той же аминокислотной последовательностью или результаты рентгенодифракционных экспериментов той же структуры с добавлением тяжелых атомов, соответственно [6]. Таким образом, решение проблемы фаз является особенно актуальной задачей белковой кристаллографии из-за отсутствия ab initio решений. 

Применение машинного обучения в области рентгеновской дифракции началось совсем недавно и сейчас бурно развивается. Чаще всего авторы пытаются обойти решение проблемы фаз, предсказывая кристаллическую структуру из доступных экспериментальных данных. Так, были разработаны модели глубокого обучения, позволяющие миновать проблему фаз, работая напрямую с функцией Паттерсона, которые получаются из дифракционных данных, не требуя информации о фазах отражений [7]. Эти модели представляют собой сверточные нейронные сети для предсказания распределения электронной плотности структуры по функции Паттерсона, демонстрируя многообещающие результаты на простых примерах дипептидов и обладают несомненным потенциалом для более широкого применения в белковой кристаллографии.
В недавней статье [8] с помощью методов глубокого обучения авторы предприняли попытку решить проблему фаз. В качестве объектов предсказания они выбрали центросимметричные структуры, для которых фазы отражений принимают только два возможных значения. Авторы представили нейронную сеть, представляющую собой бинарный классификатор из блоков трехмерных свёрток и многослойных перцептронов. Обучение проводилось на синтетических кристаллических молекулярных структурах. Этот подход продемонстрировал способность решать фазовую задачу с разрешением всего 2Å, используя всего 10-20% данных, требуемых традиционными методами. Также в работе реализована идея рециклинга фаз (phase recycling) - исходные данные прогоняются несколько раз через модель, что увеличивает точность классификации. Таким образом, впервые был продемонстрирован потенциал машинного обучения для решения проблемы фаз, но только для центросимметричных структур.

На основе методов глубокого обучения были также предложены решения аналогичной задачи в физике, особенно в области когерентной безлинзовой микроскопии. В обзорной статье [9] выделяются три подхода: DL-post-processing, где уточняются «плохие фазы», полученные из исходных интенсивностей; DL-in-processing, в котором нейронная сеть используется для непосредственного расчета фаз из интенсивностей; и DL-pre-processing, при котором обученная модель повышает разрешение микроскопического изображения, а фазы из полученного изображения определяются детерминированными методами. Однако, в кристаллографии дифракционные данные имеют гораздо меньшее разрешение, чем в когерентной микроскопии, что исключает возможность прямого применения существующих решений.
Таким образом, решение проблемы фаз белковой кристаллографии является актуальной задачей, нерешаемой рутинными методами. Инструменты на основе методов глубокого обучения способны преодолеть ограничения традиционных подходов, позволяя определять кристаллические структуры на основе ограниченных данных с низким разрешением. По мере развития исследований в этой области, вероятно, алгоритмы глубокого обучения станут незаменимым инструментом в области кристаллографии, облегчая решение сложных структур, которые ранее были неразрешимыми. Таким образом, разработка методов решения проблемы фаз с помощью ИИ является целью работы.