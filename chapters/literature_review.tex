\section{Обзор литературы}

\subsection{Рентгеновская кристаллография}


Рентгеновская кристаллография является важнейшим инструментом определения трехмерных структур кристаллов. Этот метод позволяет получить бесценные сведения об атомных и молекулярных структурах кристаллов, что крайне важно для понимания свойств и функций материалов в различных областях, включая химию, биологию и материаловедение. Рентгеноструктурный анализ основан на упругом рассеянии монохроматического рентгеновского излучения на трехмерной регулярной решетке атомов твердого вещества, что приводит к интерференции рентгеновских лучей.

Дифракция в кристалле описывается как отражения от семейств параллельных кристаллографических плоскостей элементарной ячейки кристалла. Каждая точка дифракционной картины задается набором из индексов Миллера $h,k,l$. Любое отражение имеет математическое представление, известное как структурный фактор $F(h,k,l)$, зависящий от расположения и коэффициентов рассеяния всех
атомов в элементарной ячейке кристалла. Структура кристалла получается при расчете электронной плотности с помощью Фурье-преобразования: \[\rho(x, y, z) = V^*\sum_{h,k,l} F(h,k,l) exp[(-2\pi i (hx+ky+lz)],\] где $V^*$~-- объем элементарной обратной ячейки \cite{giro}. 

\subsection{Проблема фаз}

Структурный фактор является комплексной величиной, и её амплитуда  $|F(h,k,l)|$ легко определяется по результатам обычного рентгенодифракционного эксперимента как квадратный корень из зарегистрированной интенсивности дифрагированной волны, а информация о фазе теряется. Данная проблема получила название "проблема фаз". Без знания о фазе процесс восстановления распределения электронной плотности из дифракционных данных затруднен.

Проблема фаз давно является одной из главных задач рентгеновской кристаллографии. Для ее решения были разработаны традиционные методы, такие как прямые \cite{direct} и charge-flipping \cite{charge_flipping}, но они, как правило, ограничены атомным разрешением рентгенодифракционных данных. Эти методы требуют полного набора дифракционных пиков и часто требуют выращивания высококачественных образцов кристаллов, что может быть сложной задачей, особенно для слабо рассеивающих кристаллов или больших молекул, таких как белки \cite{protein_crystallography}. Для определения структур макромолекулярных соединений разработаны методы молекулярного замещения и изоморфного замещения, требующие дополнительную информацию --- знание о структуре белка с той же аминокислотной последовательностью или результаты рентгенодифракционных экспериментов той же структуры с добавлением тяжелых атомов, соответственно \cite{acta}. Таким образом, решение проблемы фаз является особенно актуальной задачей белковой кристаллографии из-за отсутствия ab initio решений. 

Применение машинного обучения в области рентгеновской дифракции лишь недавно зародилось и сейчас бурно развивается. Чаще всего авторы пытаются обойти решение проблемы фаз, предсказывая кристаллическую структуру из доступных экспериментальных данных. Так, были разработаны модели глубокого обучения, позволяющие миновать проблему фаз, работая напрямую с функцией Паттерсона, которые получаются из дифракционных данных, не требуя информации о фазах отражений \cite{patterson}. Эти модели представляют собой сверточные нейронные сети для предсказания распределения электронной плотности структуры по функции Паттерсона, демонстрируя многообещающие результаты на простых примерах дипептидов и подчеркивая потенциал для более широкого применения в белковой кристаллографии.

В недавней статье \cite{science} с помощью методов глубокого обучения авторы предприняли попытку решить проблему фаз. В качестве объектов предсказания они выбрали центросимметричные структуры, для которых фазы отражений принимают два возможных значения --- 0 и 1. Авторы презентовали нейронную сеть, представляющую собой бинарный классификатор из блоков трехмерных свёрток и многослойных перцептронов. Обучение проводилось на синтетических кристаллических молекулярных структурах. Этот подход продемонстрировал способность решать фазовую задачу с разрешением всего 2$\text{\AA}$, используя всего 10--20$\text{\%}$ данных, требуемых традиционными методами. Также в работе реализована идея phase recycling~--- исходные данные прогоняются несколько раз через модель, что увеличивает точность классификации. Таким образом, впервые был продемонстрирован потенциал машинного обучения для решения проблемы фаз, но только для центросимметричных структур.

На основе методов глубокого обучения были также предложены решения аналогичной задачи в физике, особенно в области когерентной безлинзовой микроскопии. В обзорной статье \cite{review_physics} выделяются три подхода: DL-post-processing, где уточняются «плохие фазы», полученные из исходных интенсивностей; DL-in-processing, в котором нейронная сеть используется для непосредственного расчета фаз из интенсивностей; и DL-pre-processing, при котором обученная модель повышает разрешение микроскопического изображения, а фазы из полученного изображения определяются детерминированными методами. Однако, в кристаллографии дифракционные данные имеют гораздо меньшее разрешение, чем в когерентной микроскопии, что исключает возможность прямого применения существующих решений.

Таким образом, решение фазовой задачи белковой кристаллографии является актуальной задачей, нерешаемой рутинными методами. Инструменты на основе методов глубокого обучения способны преодолеть ограничения традиционных подходов, позволяя определять кристаллические структуры на основе ограниченных данных с низким разрешением. По мере развития исследований в этой области, вероятно, алгоритмы глубокого обучения станут незаменимым инструментом в области кристаллографии, облегчая решение сложных структур, которые ранее были неразрешимыми. Таким образом, разработка методов решения проблемы фаз с помощью ИИ является целью работы. 